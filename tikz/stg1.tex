
\newcommand{\Ra}[0]{\rightarrow}
\newcommand{\RA}[0]{\Rightarrow}
\newcommand{\Or}[0]{\ |\ }
\newcommand{\Br}[1]{\{#1\} }


%------------------------------
% category
\newcommand{\Co}{\footnotesize\textbf{;}\normalsize}
\newcommand{\CoT}{\textbf{~$;_{T}$~} }
\newcommand{\Ha}{\In{H}}
\newcommand{\CatA}{\mathcal{A}}
\newcommand{\CatB}{\mathcal{B}}
\newcommand{\cata}[1]{\llparenthesis\,#1\,\rrparenthesis}
\newcommand{\ana}[1]{[\hspace{-2.2pt}(\,#1\,)\hspace{-2.2pt}]}

\newcommand{\Alg}{\textbf{Alg}(F)} 
\newcommand{\CoAlg}{\textbf{CoAlg}(F)} 


\[
\begin{array}{rrlll}
\text{Переменные}   & x,y,f,g   &       &   &  \\
\text{Конструкторы} & C         &       &   & 
    \text{Объявлены в определениях типов}  \\
\text{Литералы}     & lit       & ::=   & i \Or d & 
    \text{Незапакованные целые} \\
 & & & & \text{или действительные числа} \\
\text{Атомы}        & a, v      & ::=   & lit \Or x & 
    \text{Аргументы функций атомарны} \\
\text{Арность функции} & k      & ::=   & \bullet   & 
    \text{Арность неизвестна} \\
                       &        & \Or   & n         & 
    \text{Арность известна } n \geq 1 \\
\\
\text{Выражения}       &   e     & ::=  & a         & \text{Атом} \\
                       &         & \Or  & f^k\ a_1 \dots a_n & 
    \text{Вызов функции } (n \geq 1) \\ 
                       &         & \Or  & \oplus\ a_1 \dots a_n & 
    \text{Вызов примитивной функции } (n \geq 1) \\  
\\
                       &         & \Or  & 
            \texttt{let}\ x\ \texttt{=}\ obj\ \texttt{in}\ e  & 
    \text{Выделение нового объекта } obj \text{ в куче} \\ 
                       &         & \Or  & 
            \texttt{case } e \texttt{ of } \{ alt_1; \dots ;alt_n \}  & 
    \text{Приведение выражения } e \text{ к СЗНФ} \\  
\\
\text{Альтернативы}    & alt     & ::=  & C\ x_1 \dots x_n \Ra e &
    \text{Сопоставление с образцом } (n \geq 1) \\
                       &         & \Or  & x \Ra e &
    \text{Альтернатива по умолчанию} \\
\\
\text{Объекты в куче}  & obj     & ::=  & FUN(x_1 \dots x_n \Ra e)  &
    \text{Функция арности } n \geq 1 \\
                        &        & \Or  & PAP(f\ a_1 \dots a_n) &
    \text{Частичное применение } f \text{ может} \\
 & & & & \text{указывать только на } FUN \\
                        &        & \Or  & CON(C\ a_1 \dots a_n) &
        \text{Полное применение конструктора } (n \geq 0) \\
                        &        & \Or  & THUNK\ e  & 
        \text{Отложенное вычисление}  \\   
                        &        & \Or  & BLACKHOLE &
        \text{Используется только во время} \\
 & & & & \text{выполнения программы} \\
\text{Программа}        & prog  & ::=   & 
    f_1 \texttt{=} obj_1\ ; \dots ;\ f_n \texttt{=} obj_n & \\

\end{array}
\]

