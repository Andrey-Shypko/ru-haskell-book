\setcounter{chapter}{17}
\chapter{Средства разработки}

В этой главе мы познакомимся с основными средствами разработки
больших программ. Мы научимся устанавливать и создавать библиотеки,
писать документацию. 

\section{Пакеты}

В Haskell есть ещё один уровень организации данных, мы
можем объединять модули в \emph{пакеты} (package). 
Также как и модули пакеты могут зависеть от других пакетов, если
они пользуются модулями их этих пакетов.
Одним пакетом мы уже пользовались и довольно часто, это
пакет \In{base}, который содержит все стандартные 
модули, например такие как \In{Prelude}, \In{Control.Applicative}
или \In{Data.Function}. 
Для создания и установки пакетов существует приложение \In{cabal}.
Оно определяет протокол организации и распространения модулей
Haskell.

\subsection{Создание пакетов}

Предположим, что мы написали программу, которая состоит 
из нескольких модулей. Пусть все модули хранятся в директории
с именем \In{src}. Для того чтобы превратить набор модулей
в пакет, нам необходимо поместить в одну директорию с \In{src}
два файла:

\begin{itemize}
\item \In{имяПакета.cabal} -- файл с описанием пакета.
\item \In{Setup.hs} -- файл с инструкциями по установке пакета
\end{itemize}

\subsubsection{.cabal}

Посмотрим на простейший файл с описанием библиотеки, этот
файл находится в одной директории с той директорией, в которой
содержатся все модули приложения и имеет расширение \In{.cabal}:

\begin{code}
Name        : Foo
Version     : 1.0

Library
  build-depends     : base
  exposed-modules   : Foo
\end{code}

Сначала идут свойства пакета. Общий формат определения свойства:

\begin{code}
ИмяСвойства : Значение
\end{code}

В примере мы указали имя пакета \In{Foo}, и версию \In{1.0}.
После того, как мы указали все свойства, мы определяем 
будет наш пакет библиотекой или исполняемой программой
или возможно он будет и тем и другим. Если пакет будет
библиотекой, то мы помещаем за набором атрибутов слово
\In{Library}, а если это исполняемая программа, то мы помещаем
слово \In{Executable}, после мы пишем описание модулей пакета, 
зависимости от других пакетов, какие модули будут видны 
пользователю. Формат составления описаний 
в этой части такой же как и в самом начале файла. 
Сначала идёт зарезервированное слово-атрибут, затем через
двоеточие следует значение. Обратите внимание на отступы за словом
\In{Library}, они обязательны и сделаны с помощью \emph{пробелов},
\In{cabal} не воспринимает табуляцию. 

Файл \In{.cabal} может содержать комментарии, они делаются
также как и в Haskell, закомментированная строка 
начинается с двойного тире.

\subsubsection{Setup.hs}

Файл \In{Setup.hs} содержит информацию о том как устанавливается
библиотека. При установке могут использоваться другие программы
и библиотеки. Пока мы будем пользоваться простейшим случаем:

\begin{code}
import Distribution.Simple
main = defaultMain
\end{code}

Этот файл позволяет нам создавать библиотеки и приложения,
которые созданы только с помощью Haskell. Это не так уж и мало!

\subsection{Создаём библиотеки}

Типичный файл \In{.cabal} для библиотеки выглядит так:

\begin{code}
Name:           pinocchio
Version:        1.1.1
Cabal-Version:  >= 1.2
License:        BSD3
License-File:   LICENSE
Author:         Mister Geppetto
Homepage:       http://pinocchio.sourceforge.net/
Category:       AI
Synopsis:       Tools for creation of woodcrafted robots
Build-Type:     Simple


Library
  Build-Depends: base
  Hs-Source-Dirs: src/  
  Exposed-modules:
    Wood.Robot.Act, Wood.Robot.Percept, Wood.Robot.Think
  Other-Modules:
    Wood.Robot.Internals
\end{code}

Этим файлом мы описали библиотеку с именем \In{pinocchio},
версия 1.1.1, она использует версию \In{cabal} не ниже 
\In{1.2}. Библиотека выпущена под лицензией BSD3. 
Файл с лицензией находится в текущей директории под
именем \In{LICENSE}. Автор библиотеки \In{Mister Geppetto}.
Подробнее узнать о библиотеке можно на её домашней странице
\In{http://pinocchio.sourceforge.net/}. Атрибут \In{Category}
указывает на широкую отрасль знаний, к которой принадлежит
наша библиотека. В данном случае мы описываем библиотеку
для построения роботов из дерева, об этом мы пишем в 
атрибуте \In{Synopsis} (краткое описание), поэтому наша библиотека
принадлежит к категории искусственный интеллект или сокращённо \In{AI}.
Последний атрибут \In{Build-Type} указывает на тип сборки
пакета. Мы будем пользоваться значением \In{Simple}, который 
соответствует сборке с помощью простейшего файла \In{Setup.hs}, 
который мы рассмотрели в предыдущем разделе. 

После описания пакета, идёт слово \In{Library}, ведь мы 
создаём библиотеку. Далее в атрибуте \In{Build-Depends}  
мы указываем зависимости для нашего пакета. Здесь мы перечисляем
все пакеты, которые мы используем в своей библиотеке. 
В данном случае мы пользовались лишь стандартной библиотекой 
\In{base}. В атрибуте \In{hs-source-dirs} мы указываем, где
искать директорию с исходным кодом библиотеки. Затем мы 
указываем три внешних модуля, они будут
доступны пользователю после установки библиотеки (атрибут 
\In{Exposed-Modules}), и внутренние скрытые модули 
(атрибут \In{Other-Modules}).

\subsection{Создаём исполняемые программы}

Типичный файл \In{.cabal} для исполняемой программы:

\begin{code}
Name:           micro
Version:        0.0
Cabal-Version:  >= 1.2
License:        BSD3
Author:         Tony Reeds
Synopsis:       Small programming language
Build-Type:     Simple

Executable micro
  Build-Depends:  base, parsec
  Main-Is:        Main.hs
  Hs-Source-Dirs: micro

Executable micro-repl
  Main-Is:        Main.hs
  Build-Depends:  base, parsec
  Hs-Source-Dirs: repl
  Other-Modules:  Utils
\end{code}

В этом файле мы описываем две программы. Компилятор 
языка и интерпретатор языка \In{micro}. Если сравнить этот
файл с файлом для библиотеки, то мы заметим лишь один новый
атрибут. Это \In{Main-Is}. Он указывает в каком модуле
содержится функция \In{main}.
После установки этого пакета будут созданы два исполняемых
файла. С именами \In{micro} и \In{micro-repl}.

\subsection{Установка пакета}

Пакеты устанавливаются с помощью команды \In{install}. 
Необходимо перейти в директорию пакета, ту, в которой
находятся два служебных файла (\In{.cabal} и \In{Setup.hs}) и 
директория с исходниками, и запустить команду:

\begin{code}
cabal install
\end{code}

Если мы нигде не ошиблись в описании пакета, не перепутали
табуляцию с пробелами при отступах, или указали без ошибок
все зависимости, то пакет успешно установится. Если это библиотека,
то мы сможем подключать экспортируемые ей модули в любом другом
модуле, просто указав их в директиве \In{import}. При этом
нам уже не важно, где находятся модули библиотеки. Мы имеем
возможность импортировать их из любого модуля. Если же пакет
был исполняемой программой, будут созданы бинарные файлы
программ. В конце \In{cabal} сообщит нам куда он их положил. 

Иногда возникают проблемы с пакетами, которые генерируют 
исполняемые файлы, а затем с их помощью устанавливают другие
пакеты. Проблема возникает из-за того, что \In{cabal} может
положить бинарный файл в директорию, которая не видна следующим
программам, которые хотят продолжить установку. В этом случае 
необходимо либо переложить созданные бинарные файлы в директорию,
которая будет им видна, или добавить директорию с новыми
бинарными файлами в \In{PATH} (под UNIX, Linux). Переменная
операционной системы PATH содержит список всех путей, в которых
система ищет исполняемые программы, если путь не указан явно.
Посмотреть содержание \In{PATH} можно, вызвав:

\begin{code}
$ echo $PATH
\end{code}

Появится строка директорий, которые записаны через двоеточие.
Для того чтобы добавить директорию \In{/data/dir} в \In{PATH} необходимо
написать: 

\begin{code}
$ PATH=$PATH:/data/dir
\end{code}

Эта команда добавит директорию в \In{PATH} для текущей
сессии в терминале, если мы хотим записать её насовсем, мы 
добавим эту команду в специальный скрытый файл \In{.bashrc}, 
он находится в домашней директории пользователя.
Под Windows добавить директорию в \In{PATH} можно с помощью
графического интерфейса. Кликните правой кнопкой мыши на иконку
\In{My Computer} (Мой Компьютер), в появившемся меню выберите вкладку
\In{Properties} (Свойства). Появится окно \In{System Properties}
(Свойства системы), в нём выберите вкладку \In{Advanced} и там 
нажмите на кнопку \In{Environment variables} (Переменные среды). 
И в этом окне будет строка \In{Path}, её мы и хотим отредактировать,
добавив необходимые нам пути.

Давайте потренируемся и создадим библиотеку и 
исполняемую программу. Создадим библиотеку, которая 
выводит на экран \In{Hello World}.
Создадим директорию \In{hello}, и в ней создадим директорию
\In{src}. Эта директория будет содержать исходный код. 
Главный модуль библиотеки экспортирует функцию приветствия:

\begin{code}
module Hello where

import Utility.Hello(hello)
import Utility.World(world)

helloWorld = hello ++ ", " ++ world ++ "!"
\end{code}

Главный модуль программы \In{Main.hs} определяет функцию
\In{main}, которая выводит текст приветствия на экран:

\begin{code}
module Main where

import Hello 

main = print helloWorld
\end{code}



У нас будет два внутренних модуля, каждый из которых
определяет синоним для одного слова. Мы поместим их
в папку  \In{Utility}. Это модуль \In{Utility.Hello}

\begin{code}
module Utility.Hello where
hello = "Hello"
\end{code}

И модуль \In{Utility.World}:

\begin{code}
module Utility.World where
world = "World"
\end{code}

Исходники готовы, теперь приступим к описанию пакета.
Создадим в корневой директории пакета файл \In{hello.cabal}. 

\begin{code}
Name:           hello
Version:        1.0
Cabal-Version:  >= 1.2
License:        BSD3
Author:         Anton
Synopsis:       Little example of cabal usage
Category:       Example
Build-Type:     Simple

Library
  Build-Depends: base == 4.*
  Hs-Source-Dirs: src/
  Exposed-modules:
    Hello
  Other-Modules:
    Utility.Hello
    Utility.World

Executable hello
  Build-Depends: base == 4.*
  Main-Is: Main.hs
  Hs-Source-Dirs: src/
\end{code}


В этом файле мы описали библиотеку и программу. 
В строке \In{base == 4.*} мы указали версию пакета \In{base}.
Запись \In{4.*} означает любая версия, которая начинается с четвёрки.
Осталось только поместить в корневую директорию пакета файл
\In{Setup.hs}. 

\begin{code}
import Distribution.Simple
main = defaultMain
\end{code}

Теперь мы можем переключиться на корневую директорию пакета 
и установить пакет:

\begin{code}
anton@anton-desktop:~/haskell-notes/code/ch-17/hello$ cabal install
Resolving dependencies...
Configuring hello-1.0...
Preprocessing library hello-1.0...
Preprocessing executables for hello-1.0...
Building hello-1.0...
[1 of 3] Compiling Utility.World    ( src/Utility/World.hs, dist/build/Utility/World.o )
[2 of 3] Compiling Utility.Hello    ( src/Utility/Hello.hs, dist/build/Utility/Hello.o )
[3 of 3] Compiling Hello            ( src/Hello.hs, dist/build/Hello.o )
Registering hello-1.0...
[1 of 4] Compiling Utility.World    ( src/Utility/World.hs, dist/build/hello/hello-tmp/Utility/World.o )
[2 of 4] Compiling Utility.Hello    ( src/Utility/Hello.hs, dist/build/hello/hello-tmp/Utility/Hello.o )
[3 of 4] Compiling Hello            ( src/Hello.hs, dist/build/hello/hello-tmp/Hello.o )
[4 of 4] Compiling Main             ( src/Main.hs, dist/build/hello/hello-tmp/Main.o )
Linking dist/build/hello/hello ...
Installing library in /home/anton/.cabal/lib/hello-1.0/ghc-7.4.1
Installing executable(s) in /home/anton/.cabal/bin
Registering hello-1.0...
\end{code}

Мы видим сообщения о процессе установки. После установки 
в текущей директории пакета появилась директория \In{dist},
в которую были помещены скомпилированные файлы библиотеки. 
В последних строках \In{cabal} сообщил нам о том, что он установил
библиотеку в директорию:

\begin{code}
Installing library in /home/anton/.cabal/lib/hello-1.0/ghc-7.4.1
\end{code}

\noindent и исполняемый файл в директорию:

\begin{code}
Installing executable(s) in /home/anton/.cabal/bin
\end{code}

С помощью различных флагов мы можем контролировать 
процесс установки пакета. Назначать дополнительные директории, 
указывать куда поместить скомпилированные файлы. 
Подробно об этом можно почитать в справке, выполнив в командной
строке одну из команд:

\begin{code}
cabal --help
cabal install --help
\end{code}

Если у вас не получилось сразу установить пакет не отчаивайтесь
и почитайте сообщения об ошибках из \In{cabal}, он информативно
жалуется о забытых зависимостях и неспособности правильно 
прочитать файл с описанием пакета.


\subsection{Удаление библиотеки}

Установленные с помощью \In{cabal} файлы видны из 
любого модуля. Имена модулей регистрируются глобально. 
Если нам захочется установить библиотеку с уже зарегистрированным 
именем, произойдёт хаос. Возможно прежняя библиотека нам уже не нужна.
Как нам удалить её? Посмотрим на решение для компилятора ghc.
Мы можем посмотреть список всех зарегистрированных в ghc библиотек
с помощью команды:

\begin{code}
$ ghc-pkg list
   Cabal-1.8.0.6
   array-0.3.0.1
   base-4.2.0.2
   ...
   ...
\end{code}

Появится длинный список с именами библиотек. Для удаления
одной из них мы можем выполнить команду:

\begin{code}
ghc-pkg unregister имя-библиотеки
\end{code}

Например так мы можем удалить только что установленную
библиотеку \In{hello}:

\begin{code}
$ ghc-pkg unregister hello
\end{code}

\subsection{Репозиторий пакетов Hackage}

Если у нас подключен интернет, то мы можем воспользоваться
наследием сообщества Haskell и установить пакет с \In{Hackage}. 
Там расположено много-много-много пакетов. Любой разработчик
Haskell может добавить свой пакет на \In{Hackage}.
Посмотреть на пакеты можно на сайте этого репозитория:

\begin{quote}
\url{http://hackage.haskell.org}
\end{quote}

Если для вашей задачи необходимо выполнить какую-нибудь
довольно общую задачу, например написать тип красно-чёрных
деревьев или построить парсер или возможно вам нужен веб-сервер,
поищите этот пакет на \In{Hackage}, он там наверняка окажется,
ещё и в нескольких вариантах.  

Для установки пакета с \In{Hackage} нужно просто написать

\begin{code}
cabal install имя-пакета
\end{code}

Возможно нам нужен очень новый пакет, который был 
только что залит автором на \In{Hackage}. Тогда выполняем:

\begin{code}
cabal update
\end{code}

Происходит обновление данных о загруженных на \In{Hackage}.
Что хорошо, вы можете
загрузить исходники из \In{Hackage}, например у вас никак 
не получается написать пакет, который устанавливался бы без
ошибок. Просто загрузим исходники какого-нибудь пакета
из \In{Hackage} и посмотрим на пример рабочего пакета.


\subsection{Дополнительные атрибуты пакета}

В файле \In{.cabal} также часто указывают такие атрибуты как:

\Desc{Maintainer}{Поле содержит адрес электронной почты 
                  тех.~поддержки}
\Desc{Stability}{Статус версии библиотеки
   (стабильная, экспериментальная, нестабильная).}
\Desc{Description}{Подробное описание назначения пакета.
                   Оно помещается на главную страницу пакета
                   в документации.}

\Desc{Extra-Source-Files}{В этом поле можно через пробел
               указать дополнительные файлы, включаемые в пакет.
               Это могут быть примеры использования, описание 
               в формате PDF или хроника изменений и другие 
               служебные файлы.}

\Desc{License-file}{Путь к файлу с лицензией.}

\smallskip

\section{Создание документации с помощью Haddock}

Если мы зайдём на Hackage, то там мы увидим длинный
список пакетов, отсортированных по категориям. 
К какой категории какой пакет относится мы указываем
в \In{.cabal}-файле в атрибуте \In{Category}. Далее рядом 
с именем пакета
мы видим краткое описание, оно берётся из атрибута \In{Synopsis}. 
Если мы зайдём 
на страницу одного из пакетов, то там мы увидим
страницу в таком же формате, что и документация
к стандартным библиотекам. Мы видим описание пакета
и ниже иерархию модулей. Мы можем зайти в заинтересовавший
нас модуль и посмотреть на объявленные функции, типы 
и классы. В самом низу страницы находится ссылка к
исходникам пакета. 

\Quote{Домашняя страница} пакета была создана с помощью
приложения \In{Haddock}. Оно генерирует документацию
в формате \In{html} по специальным комментариям.
\In{Haddock} встроен в \In{cabal}, например мы можем
сделать документацию к нашему пакету \In{hello}. 
Для этого нужно переключиться на корневую директорию
пакета и вызвать:

\begin{code}
cabal haddock
\end{code}

После этого в директории \In{dist} появится директория
\In{doc}, в которой внутри директории \In{html} находится
созданная документация. Мы можем открыть файл \In{index.html}
и там мы увидим \Quote{иерархию нашего} модуля. В модуле пока
нет ни одной функции, так получилось потому, что \In{Haddock}
помещает в документацию лишь те функции, у которых есть 
объявление типа. Если мы добавим в модуле \In{Hello.hs}:
к единственной функции объявление типа:

\begin{code}
helloWorld :: String
helloWorld = hello ++ ", " ++ world ++ "!"
\end{code}

И теперь перезапустим \In{haddock}. То мы увидим, что
в модуле  \In{Hello} появилась одна запись. 

\subsection{Комментарии к определениям}

Прокомментировать любое определение можно с помощью
комментария следующего вида:

\begin{code}
-- | Here is the comment
helloWorld :: String
helloWorld = hello ++ ", " ++ world ++ "!"
\end{code}

Обратите внимание на значок \Quote{или}, сразу после
комментариев. Этот комментарий будет включен в документацию.
Также можно писать комментарии после определения для этого
к комментарию добавляется значок степени:

\begin{code}
helloWorld :: String
helloWorld = hello ++ ", " ++ world ++ "!"
-- ^ Here is the comment
\end{code}

К сожалению на момент написания этих строк \In{Haddock}
может включать в документацию лишь латинские символы.
Комментарии могут простираться несколько строк:


\begin{code}
-- | Here is the type.
-- It contains three elements.
-- That's it.
data T = A | B | C
\end{code}

Также они могут быть блочными:

\begin{code}
{-|
   Here is the type.
   It contains three elements.
   That's it.
 -}
data T = A | B | C
\end{code}

Мы можем комментировать не только определение целиком,
но и отдельные части. Например так мы можем пояснить 
отдельные аргументы у функции:

\begin{code}
add :: Num a => a   -- ^ The first argument
             -> a   -- ^ The second argument   
             -> a   -- ^ The return value   
\end{code}

Методы класса и отдельные конструкторы типа можно комментировать
как обычные функции:

\begin{code}
data T
        -- | constructor A    
       = A      
        -- | constructor B
       | B      
        -- | constructor C
       | C     
\end{code}

Или так:

\begin{code}
data T = A      -- ^ constructor A
       | B      -- ^ constructor B
       | C      -- ^ and so on
\end{code}

Комментарии к классу:

\begin{code}
-- | С-class
class С a where
    -- | f-function
    f :: a -> a
    -- | g-function
    g :: a -> a
\end{code}

\subsection{Комментарии к модулю}

Комментарии к модулю помещаются перед объявлением имени
модуля. Эта информация попадёт в самое начало страницы документации:


\begin{code}
-- | Little example
module Hello where
\end{code}

\subsection{Структура страницы документации}

Если модуль большой, то его бывает удобно разделить 
на части, словно разделы в главе книги. Определения 
группируются по функциональности и помещаются в разные разделы
или даже подразделы.
Структура документации определяется с помощью специальных
комментариев в экспорте модуля. Посмотрим на пример:



\begin{code}
-- | Little example
module Hello(
    -- * Introduction
    -- | Here is the little example to show you
    -- how to make docs with Haddock
    
    -- * Types
    -- | The types.
    T(..),
    -- * Classes
    -- | The classes.
    C(..),
    -- * Functions
    helloWorld
    -- ** Subfunctions1
    -- ** Subfunctions2    
) where

...
\end{code}

Комментарии со звёздочкой создают раздел, а с двумя 
звёздочками -- подраздел. Те определения, которые 
экспортируются за комментариями со звёздочкой попадут в 
один раздел или подраздел. Если сразу за комментарием со 
звёздочкой идёт комментарий со знаком \Quote{или}, то
он будет помещён в самое начало раздела. В нём мы можем 
пояснить по какому принципу группируются определения
в данном разделе.

\subsection{Разметка}

С помощью специальных символов можно выделять различные 
элементы текста, например, ссылки, куски кода, названия
определений или модулей. \In{Haddock} установит 
необходимые ссылки и выделит элемент в документации.

При этом символы \texttt{/}, \texttt{'}, \texttt{`}, \texttt{"},
\texttt{@}, \texttt{<} являются специальными, если вы хотите воспользоваться
одним из специальных символов в тексте необходимо написать перед ним
обратный слэш \verb!\!. Также символы для обозначения
комментариев \texttt{*}, \texttt{|}, \texttt{\^} и \texttt{>}  являются специальными,
если они расположены в самом начале строки.


\subsubsection{Параграфы}

Параграфы определяются по пустой сроке в комментарии.
Так например мы можем разбить текст на два параграфа:

\begin{code}
-- | The first paragraph goes here.
--
-- The second paragraph goes here.
fun :: a -> b
\end{code}

\subsubsection{Блоки кода}

Существует два способа обозначения блоков кода:

\begin{code}
-- | This documentation includes two blocks of code:
--
-- @
--     f x = x + x
--     g x = x
-- @
--
-- >  g x = x * 42
\end{code}

В первом варианте мы заключаем блок кода в окружение \verb!@...@!.
Так мы можем выделить целый кусок кода. Для выделения одной строки
мы можем воспользоваться знаком \In{>}.

\subsubsection{Примеры вычисления в интерпретаторе}

В \In{Haddock} мы можем привести пример вычисления 
выражения в интерпретаторе. Это делается с помощью 
тройного символа \In{>}:

\begin{code}
-- | Two examples are given bellow:
--
-- >>> 2+3
-- 5
--
-- >>> print 1 >> print 2
-- 1
-- 2
\end{code}

Строки, которые идут сразу за строкой с символом \verb!>>>! 
помечаются как результат выполнения выражения в интерпретаторе.


\subsubsection{Имена определений}

Для того чтобы выделить имя любого определения, будь то
функция, тип или класс, необходимо заключить его в ординарные
кавычки, как в \In{'T'}. При этом \In{Haddock} установит
ссылку к определению и подсветит имя в тексте. 
Для того чтобы сослаться на определение из другого
модуля необходимо написать его полное имя, т.е.~с приставкой
имени модуля, например 
функция \In{fun}, определённая в модуле \In{M}, имеет
полное имя \In{M.fun}, тогда в комментариях мы обозначаем
её \In{'M.fun'}. 

Ординарные кавычки часто используются в английском языке
как апострофы, в таких сочетаниях как don't, isn't. 
Перед такими вхождениями ординарных кавычек можно не писать
обратный слэш. \In{Haddock} сумеет отличить их от идентификатора.

\subsubsection{Курсив и моноширинный шрифт}

Для выделения текста курсивом, он заключается в окружение
\In{...}. Для написания текста моноширинным шрифтом,
он заключается в окружение \verb!@...@!.


\subsubsection{Модули}

Для обозначения модулей используются двойные кавычки, как в 

\begin{code}
-- | This is a reference to the "Foo" module.
\end{code}

\subsubsection{Списки}

Список без нумерации обозначается с помощью звёздочек:

\begin{code}
-- | This is a bulleted list:
--
--     * first item
--
--     * second item
\end{code}

Пронумерованный список, обозначается символами \In{(n)} или \In{n.}
(\In{n} с точкой),
где \In{n} -- некоторое целое число:

\begin{code}
-- | This is an enumerated list:
--
--     (1) first item
--
--     2. second item
\end{code}

\subsubsection{Список определений}

Определения обозначаются квадратными скобками,
например комментарий:

\begin{code}
-- | This is a definition list:
--
--   [@foo@] The description of @foo@.
--
--   [@bar@] The description of @bar@.
\end{code}

\noindent в документации будет выглядеть так:

\Desc{foo}{The description of \In{foo}}.
\Desc{bar}{The description of \In{bar}}.

\smallskip

Для выделения текста моноширинным шрифтом мы воспользовались
окружением \verb!@...@!.

\subsubsection{URL}

Ссылки на сайты включаются с помощью окружения \In{<...>}.

\subsubsection{Ссылки внутри модуля}

Для того чтобы сослаться на какой-нибудь текст внутри
модуля, его необходимо отметить ссылкой. Для этого 
мы помещаем в том месте, на которое мы хотим сослаться,
запись \texttt{\#label\#}, где \In{label} -- это идентификатор 
ссылки. Теперь мы можем сослаться на это место из другого
модуля с помощью записи \texttt{"module\#label"}, где \In{module}
-- имя модуля, в котором находится ссылка \In{label}.

\section{Краткое содержание}

В этой главе мы познакомились с основными элементами
арсенала разработчика программ. Мы научились создавать 
библиотеки и документировать их.

\section{Упражнения}

Вспомните один из примеров и превратите его в библиотеку.
Например, напишите библиотеку для натуральных чисел Пеано.
