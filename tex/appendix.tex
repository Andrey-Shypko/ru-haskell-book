\addchapter{Приложения}

\addsection{Начало работы с Haskell}

\subsubsection{Компилятор}

Для программирования в Haskell нам понадобится компилятор.
Мы будем пользоваться наиболее развитым компилятором~-- GHC. 
Лучше всего устанавливать его вместе с Haskell Platform:

\begin{quote}
\url{http://hackage.haskell.org/platform/}
\end{quote}

Haskell Platform содержит стабильную версию компилятора
и много хороших, проверенных временем библиотек. Если
по каким-то причинам установить Haskell Platform не удалось.
Не отчаивайтесь, можно загрузить компилятор с сайта GHC:

\begin{quote}
\url{http://www.haskell.org/ghc/}
\end{quote}

И далее установить все необходимые библиотеки с Hackage 
с помощью \In{cabal}.

\subsubsection{Среда разработки}

Для Haskell существует очень мало сред разработки. Обычно
на Haskell программируют в каких-нибудь продвинутых текстовых
редакторах (vim, Emacs, scite, kate, notepad++). Отметим всё 
же среду разработки Leksah (\url{http://leksah.org/}), она 
написана на Haskell и её можно установить с Hackage. 
Также отметим редактор yi. Он написан на Haskell 
и его также можно установить с Hackage.

Если вы не хотите разбираться с новым текстовым редактором
или средой разработки, и вам нужна лишь подсветка синтаксиса
можно воспользоваться gedit. Пишем код в gedit, сохраняем,
переключаемся на ghci, пробуем, обновляем, пробуем, при
случае компилируем или собираем в пакет. Всё это можно делать
и в gedit.

\newpage


\addsection{Литература}

О Haskell написано много интересных книг и статей,
но все они на английском. На русском языке выходит
электронный журнал 
\Quote{Практика функционального программирования}
 (\url{fprog.ru}). Пока в нём доминируют два языка
 -- это Erlang и Haskell. 

Я бы хотел рассказать о тех книгах и статьях, которые 
мне помогли. Все они приняли активное участие в 
создании этой книги.

\addsubsection{Книги}

\begin{itemize}
\item Miran Lipova\u{c}a. Learn You A Haskell For A Great Good.
        
    Очень хорошая книга для начинающих, Haskell в картинках.
    Весёлая и познавательная книга\footnote{Обновление: книга переведена 
    на русский, вышла в издательстве ДМК Пресс}.

    \url{http://learnyouahaskell.com/}

\item Hal Daume III. Yet Another Haskell Tutorial.

    Ещё одна очень хорошая книга для начинающих. 
    Без картинок, но всё по делу. 

    \url{www.cs.utah.edu/~hal/docs/daume02yaht.pdf}

\item Paul Hudak. Haskell School of Expression.
    
    Книга, которая иллюстрирует основные принципы
    функционального программирования на примере Haskell.
    Главные достоинства -- много текста об общих 
    принципах и интересные приложения,
    картинки, музыка, анимация, управление роботами и 
    всё это на Haskell.

\item Paul Hudak. Haskell School of Music.

    Пол Хьюдак увлекается не только Haskell, но и музыкой.
    Он написал книгу, которая целиком посвящена описанию музыки в Haskell:

    \url{http://www.cs.yale.edu/homes/hudak/Papers/HSoM.pdf}

    \url{http://haskell.cs.yale.edu/}

\item Bryan O'Sullivan, Don Stewart, John Goerzen. Real World Haskell.

    Очень полезная книга в помощь тем, кто хочет научиться
    писать настоящие, серьёзные программы. Авторы подробно
    изучают вопросы, связанные с применением Haskell на практике.
    
    \url{http://book.realworldhaskell.org/}

\item Готовится к выходу к книга Сайсона Марлоу о 
    параллельных вычислениях в Haskell. Обещает
    быть очень интересной, уже известно, что книга
    будет доступна в интернете.

\end{itemize}

\addsubsection{Тематический сборник}


\subsubsection{Основы}

\begin{itemize}
\item John Hughes. Why Functional Programming Matters

\item Mark P. Jones. Functional Programming with Overloading
    and Higher-Order Polymorphism.

\item Евгений Кирпичев. Элементы функциональных 
    языков программирования, журнал Практика функционального
    программирования.

\item Paul Hudak, John Hughes, Simon Peyton Jones, Philip Wadler.
    A History of Haskell: Being Lazy With Class.

\item Simon Thompson. Programming It in Haskell.

\item Justin Bailey. Haskell Cheat Sheet.

    \url{blog.codeslower.com/static/CheatSheet.pdf}

\end{itemize}


\subsubsection{Разработка программ сверху-вниз}

\begin{itemize}
\item Дмитрий Астапов. Давно не брал я в руки шашек,
    журнал Практика Функционального программирования.
\end{itemize}


\subsubsection{Функторы и Монады}

\begin{itemize}
\item Conor McBride, Ross Paterson. Applicative programming with effects.
    Статья об аппликативных функторах.

\item Philip Wadler. The Essence of Functional Programming.
     
    Статья, в которой впервые зашла речь о применении монад
        в Haskell.

\item Tarmo Uustalu, Varmo Vene. The Essence of Dataflow Programming.
    
    Статья о комонадах, но есть много интересного и о монадах.

\item Bulat Ziganshin. Haskell I/O inside: Down the Rabbit's Hole. 
    Статья на HaskellWiki.

\item Simon Peyton Jones. Tackling the Awkward Squad:
    monadic input/output, concurrency, exceptions, and
    foreign-language calls in Haskell.
\end{itemize}

\subsubsection{Ленивые вычисления}

\begin{itemize}
\item Douglas McIlroy. Power Series, Power Serious.

\item Дмитрий Астапов. Реурсия+мемоизация=динамическое программирование,
    журнал Практика функционального программирования.

\item Сергей Зефиров. Лень бояться, журнал Практика функционального
    программирования.

\item Jerzy Karczmarczuk. Specific “scientific” data structures, 
    and their processing.



\end{itemize}

\subsubsection{Структурная рекурсия}

\begin{itemize}
\item Graham Hutton. A tutorial on the universality
    and expressiveness of fold

\item Jeremy Gibbons. Origami Programming.

\item Jeremy Gibbons, Geraint Jones. The Under-Appreciated Unfold. 

\end{itemize}

\subsubsection{Лямбда-исчисление и функциональное программирование}

\begin{itemize}
\item Шалак В.И. Шейнфинкель и комбинаторная логика.
    
\item Paul Hudak: Conception, Evolution, and Application 
        of Functional Programming Languages.

      Длинная статья о развитии функциональных языков.
      Там есть главы о лямбда-исчислении. 

\item Бенджамин Пирс. Типы в языках программирования.

     Большая книга о теории типов. 

     \url{http://newstar.rinet.ru/~goga/tapl/}

\item Денис Москвин. Системы типизации лямбда-исчисления. 

    Курс видео-лекций.

    \url{http://www.lektorium.tv/course/?id=22797}

\item John Harrison. Introduction to Functional Programming.

    Курс лекций по функциональному программированию, который
    читался в Университете Кэмбридж.

\item А. Филд, П. Харрисон, Функциональное программирование,
    Москва \Quote{Мир}, 1993. 

    Большая книга для читателей, всерьёз заинтересовавшихся
    функциональным программированием. Прочитав её, вы 
    сможете не только пользоваться ФП-языками но и написать 
    такой язык самостоятельно.

\item Rinus Plasmeijer and Marko van Eekelen. Functional Programming
    and Parallel Graph Rewriting.

    Также глубокая книга по ФП. Исследуются вопросы распараллеливания
    функциональных программ, построение компиляторов. 


\end{itemize}

\subsubsection{Теория категорий}

Две очень хорошие книги для начинающих:

\begin{itemize}
\item Maarten M. Fokkinga. Gentle Introduction to Category Theory.

\url{wwwhome.cs.utwente.nl/~fokkinga/mmf92b.pdf}

\item Steve Awodey. Category Theory.
    
\end{itemize}

\begin{itemize}

\item Eugenia Cheng, Simon Willerton aka TheCatsters. Курс видео-лекций
    на youtube. 

    \url{http://www.scss.tcd.ie/Edsko.de.Vries/ct/catsters/linear.php}
    \url{http://www.youtube.com/user/TheCatsters}
\end{itemize}

Статьи по категориальным типам:

\begin{itemize}

\item Varmo Vene. Categorical Programming 
    with Inductive and Coinductive Types. Phd-диссертация.
    
\item Erik Meijer, Graham Hutton.
    Bananas in Space: Extending Fold and Unfold to Exponential Types.

\item Martin Erwig. Categorical Programming with Abstract Data Types.

\item Martin Erwig. Metamorphic Programming: Structured Recursion 
    for Abstract Data Types.

\end{itemize}

\subsubsection{Практика}

\begin{itemize}
\item Conal Elliott. Denotational design with type class morphisms.

\item Johan Tibell. High Performance Haskell. Слайды с выступления. 

\item Simon Marlow. Parallel and Concurrent Programming in Haskell.

\item Edward Z. Yang. Блог о Haskell в картинках. Много
    полезной информации о лени и устройстве ghc.
    \url{http://blog.ezyang.com/about/}

\item Oleg Kiselyov. Блог в том числе и о Haskell. Много решений интересных
    и нетривиальных задач. 
\url{http://okmij.org/ftp/}

\end{itemize}

\subsection{Как работает GHC}

\begin{itemize}
\item Документация GHC:
\url{http://hackage.haskell.org/trac/ghc/wiki/Commentary}

\item Don Stewart. Multi-paradigm Just-In-Time Compilation. BS Thesis, 2002.

    Автор пробует компилировать Haskell-код в Java-код. При этом
    очень доступно объясняеются внутренности STG. 

\item Simon Marlow, Simon Peyton Jones. The Glasgow Haskell Compiler. 
    The Architecture of Open Source Application, Volume 2, 2012.

\item Simon Marlow, Simon Peyton Jones. Making a Fast Curry: Push/Enter vs.
    Eval/Apply for Higher-order Languages. ICFP'04.

\item Simon Peyton Jones. Implementing lazy functional languages
on stock hardware: the Spineless Tagless G-machine.

\item Simon Marlow, Tim Harris, Roshan P. James, Simon Peyton Jones.
    Parallel Generational-Copying Garbage Collection with a 
    Block-Structured Heap. ISMM'08.

\item Simon Peyton Jones, Andre Santos. 
    A transformation-based optimizer for Haskell.
    Science of computer programming, 1998.

\item Simon Peyton Jones, John Launchbury. 
    Unboxed values as first citizens in a non-strict 
    functional programming language. 1991.

\item Simon Marlow, Simon Peyton Jones. 
    Secrets of Glasgow Haskell Compiler inliner. 1999

    Статья о тонкостях реализации прагмы \In{INLINE}.

\item Simon Peyton Jones, Andrew Tolmach, Tony Hoare.
    Playing by the Rules, ICFP 2001

    Статья о прагме \In{RULES}.
\end{itemize}


\addsubsection{И все-все-все}

Если вдруг у вас возникли вопросы по Haskell, и рядом с вами не оказалось
того, кто мог бы на них ответить, и в книгах нет ответа, вы 
можете спросить у сообщества Haskell,
в haskell-cafe, там вам быстро и с радостью ответят:

\begin{quote}
\url{http://www.haskell.org/mailman/listinfo/haskell-cafe}
\end{quote}

Сообщество Haskell славится радушием и терпимостью 
к начинающим. Там много информации о выпусках новых библиотек,
конференциях, обучающих программах и просто разговоры о том-о-сём.

Также стоит отметить журнал \emph{Monad.Reader}:

\begin{quote}
\url{http://themonadreader.wordpress.com/}
\end{quote}

\newpage

\addsection{Обзор Hackage}

Число пакетов, загруженных на Hackage, уже перевалило за 
2000. В Hackage легко заблудиться. 
Очень часто не разберёшься какой из пакетов
выбрать. К тому же многие из них заброшены или просто 
не подходят для использования в серьёзных приложениях. 
Но среди них есть и очень хорошие пакеты. Некоторые
из них включены в \In{Haskell Platform}. 
Ниже приведён тематический обзор наиболее популярных пакетов.

\addsubsection{Стандартные библиотеки}

Все приведённые в этом подразделе библиотеки включены
в \In{Haskell Platform}.

Полный список библиотек для \In{Haskell Platform}
можно посмотреть на сайте \url{http://lambda.haskell.org/hp-tmp/docs}.

\begin{itemize}
\item \textbf{Начало-всех-начал}: \In{base}

    Библиотека включает в себя все стандартные определения,
    например модули \In{Prelude}, \In{Data.List}, \In{Control.Monad}
    и многие другие.

\item \textbf{Стандартные монады}:  \In{mtl}

    Включает монады \In{State}, \In{Writer}, \In{Reader}
    и другие.
    

\item \textbf{Контейнеры}: \In{containers}

    Ассоциативные массивы, множества, последовательности, деревья. 

\item \textbf{Массивы}: \In{array}

\item \textbf{Графы}: \In{fgl}

\item \textbf{Архиваторы}: \In{zlib}
\item \textbf{Вычисление по значению}: \In{deepseq}
    
    Обычная функция \In{seq}, позволяет привести данное
    выражение к слабой заголовочной нормальной форме,
    если нам всё же необходимо вычислить значение полностью,
    мы можем воспользоваться функцией \In{deepseq} из 
    одноимённой библиотеки.

\item \textbf{Параллельное программирование}: \In{stm} и \In{parallel}

\item \textbf{Временная арифметика, календарь}: \In{time}
\item \textbf{Парсинг}: \In{parsec}
\item \textbf{Регулярные выражения}: \In{regex-base}, \In{regex-posix}
\item \textbf{Построение структурированного текста}: \In{pretty}
\item \textbf{Тестирование программ}: \In{HUnit}, \In{QuickCheck}
\item \textbf{Управление файловой системой}: \In{directory}
\item \textbf{Работа с путями к файлам/директориям}: \In{filepath}
\item \textbf{Сетевые библиотеки}: \In{network}, \In{HTTP}, \In{cgi}.
\item \textbf{3д Графика}: \In{OpenGL}, \In{GLUT}.

\item \textbf{Монадные трансформеры}: \In{transformers}

    Мы не коснулись этой темы, но вот краткое пояснение:
    монадные трансформеры позволяют комбинировать несколько
    монад. Например, если нам нужно использовать чтение-запись 
    в файл совместно с изменяемым состоянием.

\end{itemize}

\addsubsection{Эффективные типы данных}

\begin{itemize}
\item \textbf{Списки}: {dlist} -- эффективное объединение списков.

            Если вы часто пользуетесь операцией \In{++},
            то необходимо заботиться о том, чтобы скобки
            всегда группировались вправо. Как в \In{a++(b++(c++d))}.
            Иначе время объединения из линейного превратится
            в квадратичное. Библиотека \In{dlist} предоставляет
            специальный тип списков, для которых не важно как 
            группируются скобки при объединении. Время 
            объединения всегда будет линейным. 

\item \textbf{Строки}: \In{bytestring}
    
    Если ваша программа загружена обработкой строк,
    и работает слишком медленно, рассмотрите вариант
    перехода со стандартных строк на тип \In{ByteString},
    это может на порядок увеличить быстродействие.

\item \textbf{Текст}: \In{text} или \In{utf8-string} 

    Работа с текстом в формате Unicode. Часто проблемы 
    возникают при необходимости обработки русского текста
    закодированного в Unicode. Для решения этой проблемы 
    можно воспользоваться одной из этих библиотек.

\item \textbf{Двоичные данные}:  
       \In{binary} или \In{cereal} -- Сериализация/десериализация данных.

\item \textbf{Случайные числа}: \In{mersenne-random-pure64}

    Эффективный генератор случайных чисел.

\item \textbf{Ввод-вывод}: \In{iteratee}

    Эффективная реализация ввода-вывода. Если вам нужно читать 
    или писать данные из большого числа файлов, эта библиотека 
    может существенно помочь.

\item \textbf{Контейнеры}: \In{unordered-containers}

    Альтернатива стандартной библиотеке \In{containers}. 
    Эффективные типы \In{Map} и \In{Set}.  

\item \textbf{Последовательности}: \In{fingertree}

    Используются для работы с очередями различного типа.

\item \textbf{Массивы}: \In{vector}

    Эффективный тип для представления массивов. Замена
    стандартному типу \In{Data.Array}.

\item \textbf{Матрицы}: \In{hmatrix}, \In{repa}

\end{itemize}

\addsubsection{Разработка программ}

\begin{itemize}

\item Тестирование, проверка инвариантов: \In{QuickCheck}

\item Оценка быстродействия: \In{criterion}

\item Просмотр Core в человеческом виде: \In{ghc-core}

\item Настройка сборки мусора: \In{ghc-gc-tune}

\item Трассировка программ: \In{hat}

\end{itemize}

\addsubsection{И все-все-все}

\begin{itemize}

\item \textbf{Парсинг}: \In{parsec} или \In{attoparsec}
\item \textbf{Языки разметки}: \In{pandoc}, \In{xhtml}, \In{tagsoup}, 
        \In{blaze-html}, \In{html}
\item \textbf{XML}: \In{xml}, \In{HaXml}
\item \textbf{JSON}: \In{json}, \In{aeson}
\item \textbf{Web}: \In{happstack}, \In{snap}, \In{yesod}, \In{hakyll}
\item \textbf{Сетевые библиотеки}: \In{network}, \In{HTTP}, \In{cgi}, \In{curl}

\item \textbf{Графика}: \In{diagrams}, \In{gnuplot}, \In{SDL}
\item \textbf{3д графика}: \In{OpenGL}, \In{GLFW}, \In{GLUT} 

\item \textbf{Базы данных}: \In{HDBC}
\item \textbf{Встраиваемые приложения реального времени 
    с жёсткими ограничениями}: \In{atom}

\item \textbf{GUI}: \In{wxHaskell}, \In{gtk2hs}

\item \textbf{Оценка производительности программ}: \In{criterion}

\item \textbf{Статистика}: \In{statistics}

\item \textbf{Парсинг и генерация кода Haskell}: haskell-src-exts

\item \textbf{FRP}: \In{reactive}, \In{reactive-banana}, \In{yampa}

\item \textbf{Линейная алгебра}: \In{vector-space}, \In{hmatrix}

\end{itemize}

\newpage
\addsection{Места}

Где культивируется Haskell? 

\addsubsection{Университеты}

Посмотрим на университеты, в которых 
Haskell преподают, развивают и применяют:

\begin{itemize}
\item Британия: Эдинбург, Ноттингем, Оксфорд (лаборатория информатики),
    Глазго.

\item Америка: Йельский, Коннектикут, Техас, Оклахома, Портлэнд

\item Нидерланды: Утрехт

\item Швеция: Технологический Чалмерса, Гёттинген. 

\item Австралия: Новый Южный Уэльс, Западной Австралии

\item и другие, полный список на 
\url{http://www.haskell.org/haskellwiki/Haskell_in_education}. 

\end{itemize}

\addsubsection{Компании}

\begin{itemize}
\item Microsoft Research -- разрабатывают GHC.

\item Galios -- ведут исследования и решают практические
задачи на ФП-языках, особенно на Haskell.

\item Well-Typed -- решают практические задачи, консультируют
    и всё на Haskell. Также занимаются организацией Haskell-слётов,
    поддержкой стандартных библиотек. 

\item и другие, полный список на 
\url{http://www.haskell.org/haskellwiki/Haskell_in_industry}

\end{itemize}


