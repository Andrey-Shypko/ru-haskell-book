\setcounter{chapter}{9}
\chapter{Структурная рекурсия}

Структурная рекурсия определяет способ построения и
преобразования значений по виду типа (по составу его конструкторов).
Функции, которые преобразуют значения 
мы будем называть \emph{свёрткой} (fold),
а функции которые строят значения -- 
\emph{развёрткой} (unfold). 
Эта рекурсия встречается очень 
часто, мы уже пользовались ею и не раз, но в этой главе
мы остановимся на ней поподробнее.

\section{Свёртка}

\Structs{свёртка}
Свёртку значения можно представить как процесс, который
заменяет в дереве значения все конструкторы на подходящие
по типу функции. 

\subsection{Логические значения}

Вспомним определение логических значений:

\begin{code}
data Bool = True | False
\end{code}

У нас есть два конструктора-константы. Любое значение
типа \In{Bool} может состоять либо из одного конструктора
\In{True}, либо из одного конструктора \In{False}. 
Функция свёртки в данном случае принимает две константы
одинакового типа \In{a} и возвращает функцию, которая превращает
значение типа \In{Bool} в значение \mbox{типа \In{a}}, 
заменяя конструкторы на переданные значения:

\begin{code}
foldBool :: a -> a -> Bool -> a
foldBool true false = \b -> case b of
    True    -> true
    False   -> false
\end{code}

Мы написали эту функцию в композиционном стиле для 
того, чтобы подчеркнуть, что функция преобразует 
значение типа \In{Bool}. Определим несколько знакомых
функций через функцию свёртки, начнём с отрицания:

\begin{code}
not :: Bool -> Bool
not = foldNat False True
\end{code}

Мы поменяли конструкторы местами, если на вход поступит
\In{True}, то мы вернём \In{False} и наоборот. 
Теперь посмотрим на \Quote{и} и \Quote{или}:

\begin{code}
(||), (&&) :: Bool -> Bool -> Bool

(||) = foldNat  (const True)  id
(&&) = foldNat  id            (const False)
\end{code}

Определение функций \Quote{и} и \Quote{или} через свёртки
подчёркивает, что они являются взаимно обратными.
Смотрите, эти функции принимают значение типа \In{Bool}
и возвращают функцию \In{Bool -> Bool}. 
Фактически функция свёртки для \In{Bool} является 
\In{if}-выражением, только в этот раз мы пишем условие
в конце. 

\subsection{Натуральные числа}

У нас был тип для натуральных чисел Пеано:

\begin{code}
data Nat = Zero | Succ Nat
\end{code}

Помните мы когда-то записывали определения типов в стиле классов:

\begin{code}
data Nat where
    Zero :: Nat
    Succ :: Nat -> Nat
\end{code}

Если мы заменим конструктор \In{Zero} на значение типа \In{a},
то конструктор \In{Succ} нам придётся заменять на функцию
типа \In{a -> a}, иначе мы не пройдём проверку типов. 
Представим, что \In{Nat} это класс:

\begin{code}
data Nat a where
    zero :: a
    succ :: a -> a
\end{code}

Из этого определения следует функция свёртки:

\begin{code}
foldNat :: a -> (a -> a) -> (Nat -> a)
foldNat zero succ = \n -> case n of
    Zero    -> zero
    Succ m  -> succ (foldNat zero succ m)
\end{code}

Обратите внимание на рекурсивный вызов функции \In{foldNat}
мы обходим всё дерево значения, заменяя каждый конструктор. 
Определим знакомые функции через свёртку:

\begin{code}
isZero :: Nat -> Bool
isZero = foldNat True (const False)
\end{code}

Посмотрим как вычисляется эта функция:

\begin{code}
        isZero Zero
=>      True            -- заменили конструктор Zero

        isZero (Succ (Succ (Succ Zero)))
=>      const False (const False (const False True))
                        -- заменили и Zero и Succ
=>      False
\end{code}

Что интересно за счёт ленивых вычислений на самом деле
во втором выражении произойдёт лишь одна замена.
Мы не обходим всё дерево, нам это и не нужно, а смотрим
лишь на первый конструктор, если там \In{Succ}, то
произойдёт замена на постоянную функцию, которая игнорирует
свой второй аргумент и рекурсивного вызова функции свёртки
не произойдёт, совсем как в исходном определении! 


\begin{code}
even, odd :: Nat -> Bool

even    = foldNat True  not
odd     = foldNat False not
\end{code}

Эти функции определяют чётность числа, сдесь
мы пользуемся тем свойством, что \In{not (not a) == a}.

Определим сложение и умножение:

\begin{code}
add, mul :: Nat -> Nat -> Nat

add a  = foldNat a     Succ
mul a  = foldNat Zero  (add a)
\end{code}


\subsection{Maybe}

Вспомним определение типа для результата частично
определённых функций:

\begin{code}
data Maybe a = Nothing | Just a
\end{code}

Перепишем словно это класс:

\begin{code}
data Maybe a b where
    Nothing :: b
    Just    :: a -> b
\end{code}

Этот класс принимает два параметра, поскольку исходный
тип \In{Maybe} принимает один. Теперь несложно догадаться
как будет выглядеть функция свёртки, мы просто получим
стандартную функцию \In{maybe}. 
Дадим определение экземпляра функтора и монады через свёртку:

\begin{code}
instance Functor Maybe where
    fmap f = maybe Nothing (Just . f)

instance Monad Maybe where
    return      = Just
    ma >>= mf   = maybe Nothing mf ma
\end{code}

\subsection{Списки}

Функция свёртки для списков это функция \In{foldr}. Выведем
её из определения типа:

\begin{code}
data [a] = a : [a] | []
\end{code}

Представим, что это класс:

\begin{code}
class [a] b where
    cons    :: a -> b -> b
    nil     :: b
\end{code}

Теперь получить определение для \In{foldr} совсем просто:

\begin{code}
foldr :: (a -> b -> b) -> b -> [a] -> b
foldr cons nil = \x -> case x of
    a:as    -> a `cons` foldr cons nil as
    []      -> nil    
\end{code}

Мы обходим дерево значения, заменяя конструкторы методами
нашего воображаемого класса. Определим несколько стандартных
функций для списков через свёртку.

Первый элемент списка:

\begin{code}
head :: [a] -> a
head = foldr const (error "empty list")
\end{code}

Объединение списков:

\begin{code}
(++) :: [a] -> [a] -> [a]
a ++ b = foldr (:) b a
\end{code}

В этой функции мы реконструируем заново первый список
но в самом конце заменяем пустой список в хвосте \In{a}
на второй аргумент, так и получается объединение списков.
Обратите внимание на эту особенность, скорость выполнения
операции \In{(++)} зависит от длины первого списка. 
Поэтому между двумя выражениями 

\begin{code}
((a ++ b) ++ c) ++ d
a ++ (b ++ (c ++ d))
\end{code}

Нет разницы в итоговом результате, но есть огромная разница
по скорости вычисления! Второй гораздо быстрее. Убедитесь
в этом! Реализуем объединение списка списков
в один список:

\begin{code}
concat :: [[a]] -> [a]
concat = foldr (++) []
\end{code}

Через свёртку можно реализовать и функцию преобразования списков:

\begin{code}
map :: (a -> b) -> [a] -> [b]
map f = foldr ((:) . f) []
\end{code}

Если смысл выражения \In{((:) . f)} не совсем понятен, давайте
распишем его типы:

\begin{diagram}
\texttt{a} && \rTo^{\texttt{f}} && \texttt{b} && \rTo^{\texttt{(:)}}& && \texttt{([b] -> [b])} \\
\end{diagram}

Напишем функцию фильтрации:

\begin{code}
filter :: (a -> Bool) -> [a] -> [a]
filter p = foldr (\a as -> foldBool (a:as) as (p a)) []
\end{code}

Тут у нас целых две функции свёртки. Если значение
предиката \In{p} истинно, то мы вернём все элементы списка,
а если ложно отбросим первый элемент.
Через \In{foldr} можно даже определить функцию с хвостовой
рекурсией \In{foldl}. Но это не так просто. Всё же попробуем.
Для этого вспомним определение:

\begin{code}
foldl :: (a -> b -> a) -> a -> [b] -> a
foldl f s []        = s
foldl f s (a:as)    = foldl f (f s a) as
\end{code}

Нам нужно привести это определение к виду \In{foldr}, 
нам нужно выделить два метода воображаемого класса списка
\In{cons} и \In{nil}:

\begin{code}
foldr :: (a -> b -> b) -> b -> [a] -> b
foldr cons nil = \x -> case x of
    a:as    -> a `cons` foldr cons nil as
    []      -> nil    
\end{code}

Перенесём два последних аргумента определения
\In{foldl} в правую часть, воспользуемся
лямбда-функциями и \In{case}-выражением:

\begin{code}
foldl :: (a -> b -> a) -> [b] -> a -> a
foldl f = \x -> case x of    
    []      -> \s -> s
    a:as    -> \s -> foldl f as (f s a)
\end{code}

Мы поменяли местами порядок следования аргументов
(второго и третьего). Выделим тождественную функцию 
в первом уравнении \In{case}-выражения и функцию
композиции во втором. 


\begin{code}
foldl :: (a -> b -> a) -> [b] -> a -> a
foldl f = \x -> case x of    
    []      -> id
    a:as    -> foldl f as . (flip f a)
\end{code}

Теперь выделим функции \In{cons} и \In{nil}:

\begin{code}
foldl :: (a -> b -> a) -> [b] -> a -> a
foldl f = \x -> case x of    
    []      -> nil 
    a:as    -> a `cons` foldl f as 
    where nil   = id
          cons  = \a b -> b . flip f a
                = \a   -> ( . flip f a)
\end{code}

Теперь запишем через \In{foldr}:

\begin{code}
foldl :: (a -> b -> a) -> a -> [b] -> a
foldl f s xs = foldr (\a -> ( . flip f a)) id xs s
\end{code}

Кажется мы ошиблись в аргументах, ведь \In{foldr}
принимает три аргумента. Дело в том, что в функции
\In{foldr} мы сворачиваем списки в функции, последний
аргумент предназначен как раз для результирующей функции.
Отметим, что из определения можно исключить два последних
аргумента с помощью функции \In{flip}. 

\subsubsection{Вычислительные особенности foldl и foldr}

Если посмотреть на выражение, которое получается в 
результате вычисления \In{foldr} и \In{foldl} можно
понять почему они так называются. 

В левой свёртке \In{foldl} скобки группируются влево, поэтому 
на конце \In{l} (left):

\begin{code}
foldl f s [a1, a2, a3, a4] =
    (((s `f` a1) `f` a2) `f` a3) `f` a4
\end{code}

В правой свёртке \In{foldr} скобки группируются вправо, 
поэтому на конце \In{r} (right):

\begin{code}
foldr f s [a1, a2, a3, a4]
    a1 `f` (a2 `f` (a3 `f` (a4 `f` s)))
\end{code}

Кажется, что если функция \In{f} ассоциативна 

\begin{code}
(a `f` b) `f` c  = a `f` (b `f` c)
\end{code}


\noindent то нет разницы какую свёртку применять. 
Разницы нет по смыслу, но может быть существенная разница 
в скорости вычисления. Рассмотрим функцию \In{concat}, ниже 
два определения:

\begin{code}
concat  = foldl (++) []
concat  = foldr (++) []
\end{code}

Какое выбрать? Результат и в том и в другом случае 
одинаковый (функция \In{++} ассоциативна). Стоит
выбрать вариант с правой свёрткой. В первом варианте
скобки будут группироваться влево, это чудовищно скажется
на производительности. Особенно если в конце небольшие
списки:  

\begin{code}
Prelude> let concatl  = foldl (++) []
Prelude> let concatr  = foldr (++) []
Prelude> let x = [1 .. 1000000]
Prelude> let xs = [x,x,x] ++ map return x
\end{code}

Последним выражением мы создали список списков, 
в котором три списка по миллиону элементов, а в конце
миллион списков по одному элементу. Теперь попробуйте
выполнить \In{concatl} и \In{concatr} на списке \In{xs}.
Вы заметите разницу по скорости печати. Также для 
сравнения можно установить флаг: \In{:set +s}.

Также интересной особенностью \In{foldr} является
тот факт, что за счёт ленивых вычислений \In{foldr}
не нужно знать весь список, правая свёртка может работать 
и на бесконечных списках, в то время как \In{foldl} 
не вернёт результат, пока не составит всё выражение.
Например такое выражение будет вычислено:

\begin{code}
Prelude> foldr (&&) undefined $ True : True : repeat False
False
\end{code}

За счёт ленивых вычислений мы отбросили оставшуюся
(бесконечную) часть списка. 
По этим примерам может показаться, что левая свёртка такая 
не нужна совсем, но не все операции ассоциативны. Иногда полезно собирать 
результат в обратном порядке, например так в \In{Prelude}
определена функция \In{reverse}, которая переворачивает
список:

\begin{code}
reverse :: [a] -> [a]
reverse = foldl (flip (:)) []
\end{code}

\subsection{Деревья}

Мы можем определить свёртку и для деревьев. Вспомним тип:

\begin{code}
data Tree a = Node a [Tree a]
\end{code}

Запишем в виде класса:

\begin{code}
data Tree a b where
    node :: a -> [b] -> b
\end{code}

В этом случае есть одна тонкость.
У нас два рекурсивных типа: само дерево и 
внутри него -- список. Для преобразования списка
мы воспользуемся функцией \In{map}:

\begin{code}
foldTree :: (a -> [b] -> b) -> Tree a -> b
foldTree node = \x -> case x of
    Node a as -> node a (map (foldTree node) as)
\end{code}

Найдём список всех меток:

\begin{code}
labels :: Tree a -> [a]
labels = foldTree $ \a bs -> a : concat bs
\end{code}

Мы объединяем все метки из поддеревьев в один список
и присоединяем к нему метку из текущего узла.

Сделаем дерево экземпляром класса \In{Functor}:

\begin{code}
instance Functor Tree where
    fmap f = foldTree (Node . f)
\end{code}

Очень похоже на \In{map} для списков. Вычислим глубину
дерева:

\begin{code}
depth :: Tree a -> Int
depth = foldTree $ \a bs 0 -> 1 + foldr max 0 bs
\end{code}

В этой функции за каждый узел мы прибавляем к результату
единицу, а в списке находим максимум среди всех поддеревьев. 

\section{Развёртка}

\Structs{развёртка}
С помощью развёртки мы постепенно извлекаем значение
рекурсивного типа из значения какого-нибудь другого типа.
Этот процесс очень похож на процесс вычисления по имени.
Сначала у нас есть отложенное вычисление \verb!#! или thunk. 
Затем мы применяем к нему функцию редукции и у нас появляется 
корневой конструктор.
А в аргументах конструктора снова сидят \verb!#!. Мы применяем
редукцию к ним. И так пока не \Quote{развернём} всё значение.

\subsection{Списки}

Для разворачивания списков в \In{Data.List} есть специальная
функция \In{unfoldr}. Присмотримся сначала к её типу:

\begin{code}
unfoldr :: (b -> Maybe (a, b)) -> b -> [a]
\end{code}

Функция развёртки принимает стартовый элемент, а возвращает
значение типа пары от \In{Maybe}. Типом \In{Maybe} мы кодируем 
конструкторы списка:

\begin{code}
data [a] b where
    (:)  :: a -> b -> b     -- Maybe (a, b)
    []   :: b               -- Nothing
\end{code}

Конструктор пустого списка не нуждается в аргументах, поэтому
его мы кодируем константой \In{Nothing}. Объединение принимает
два аргумента голову и хвост, поэтому \In{Maybe} содержит пару
из головы и следующего элемента для разворачивания.
Закодируем это определение:

\begin{code}
unfoldr :: (b -> Maybe (a, b)) -> b -> [a]
unfoldr f = \b -> case (f b) of
    Just (a, b') -> a : unfoldr f b'
    Nothing      -> []
\end{code}

Или мы можем записать это более кратко с помощью свёртки \In{maybe}:

\begin{code}
unfoldr :: (b -> Maybe (a, b)) -> b -> [a]
unfoldr f = maybe [] (\(a, b) -> a : unfoldr f b)
\end{code}

Смотрите, перед нами коробочка (типа \In{b}) с подарком (типа \In{a}), 
мы разворачиваем, а там пара: подарок (типа \In{a}) и ещё одна коробочка. Тогда
мы начинаем разворачивать следующую коробочку и так далее по цепочке,
пока мы не развернём не обнаружим \In{Nothing}, это означает
что подарки кончились.


Типичный пример развёртки это функция \In{iterate}. 
У нас есть стартовое значение типа \In{a} и функция
получения следующего элемента \In{a -> a}

\begin{code}
iterate :: (a -> a) -> a -> [a]
iterate f = unfoldr $ \s -> Just (s, f s)
\end{code}

Поскольку \In{Nothing} не используется цепочка подарков 
никогда не оборвётся. Если только нам не будет лень их
разворачивать. Ещё один характерный пример это 
функция \In{zip}:

\begin{code}
zip :: [a] -> [b] -> [(a, b)]
zip = curry $ unfoldr $ \x -> case x of
    ([]     , _)     -> Nothing
    (_      ,[])     -> Nothing
    (a:as   , b:bs)  -> Just ((a, b), (as, bs)) 
\end{code}

Если один из списков обрывается, то прекращаем разворачивать.
А если оба содержат голову и хвост, то мы помещаем в голову списка
пару голов, а в следующий элемент для разворачивания пару хвостов.

\subsection{Потоки}

Для развёртки хорошо подходят типы у которых, всего один
конструктор. Тогда нам не нужно кодировать альтернативы.
Например рассмотрим потоки:

\begin{code}
data Stream a = a :& Stream a
\end{code}

Они такие же как и списки, только без конструктора пустого списка.
Функция развёртки для потоков имеет вид:

\begin{code}
unfoldStream :: (b -> (a, b)) -> b -> Stream a
unfoldStream f  = \b -> case f b of
    (a, b') -> a :& unfoldStream f b'
\end{code}

И нам не нужно пользоваться \In{Maybe}. 
Напишем функции генерации потоков:

\begin{code}
iterate :: (a -> a) -> a -> Stream a
iterate f = unfoldStream $ \a -> (a, f a)

repeat :: a -> Stream a
repeat = unfoldStream $ \a -> (a, a)

zip :: Stream a -> Stream b -> Stream (a, b)
zip = curry $ unfoldStream $ \(a :& as, b :& bs) -> ((a, b), (as, bs))
\end{code}

\subsection{Натуральные числа}

Если присмотреться к натуральным числам, то можно
заметить, что они очень похожи на списки. Списки без элементов.
Это отражается на функции развёртки. Для натуральных
чисел мы будем возвращать не пару а просто слкедующий
элемент для развёртки:

\begin{code}
unfoldNat :: (a -> Maybe a) -> a -> Nat
unfoldNat f = maybe Zero (Succ . unfoldNat f)
\end{code}

Напишем функцию преобразования из целых чисел в натуральные:


\begin{code}
fromInt :: Int -> Nat
fromInt = unfoldNat f
    where f n
            | n == 0    = Nothing
            | n >  0    = Just (n-1)
            | otherwise = error "negative number"
\end{code}

Обратите внимание на то, что в этом определении не
участвуют конструкторы для \In{Nat}, хотя мы и строим
значение типа \In{Nat}. Конструкторы для \In{Nat} 
как и в случае списков кодируются типом \In{Maybe}.
Развёртка используется гораздо реже свёртки. 
Возможно это объясняется необходимостью кодирования 
типа результата некоторым промежуточным типом. 
Определения теряют в наглядности. Смотрим на функцию,
а там \In{Maybe} и не сразу понятно \emph{что} мы строим:
натуральные числа, списки или ещё что-то.

\section{Краткое содержание}

В этой главе мы познакомились с особым видом рекурсии. 
Мы познакомились со структурной рекурсией. Типы определяют
не только значения, но и способы их обработки. Структурная
рекурсия может быть выведена из определения типа. 
Есть языки программирования, в которых мы определяем тип 
и получаем функции структурной рекурсии в подарок. 
Есть языки, в которых структурная рекурсия является единственным
возможным способом составления рекурсивных функций. 

Обратите внимание на то, что в этой главе мы определяли
рекурсивные функции, но рекурсия встречалась лишь 
в определении для функции свёртки и развёртки. 
Все остальные функции не содержали рекурсии, более того
почти все они определялись в бесточечном стиле.
Структурная рекурсия это своего рода комбинатор 
неподвижной точки, но не общий, а специфический для
данного рекурсивного типа.

Структурная рекурсия бывает свёрткой и развёрткой. 

\begin{itemize}

\item \emph{Cвёрткой} (fold) мы получаем значение некоторого 
произвольного типа из данного рекурсивного типа. 
При этом все конструкторы заменяются на функции, 
которые возвращают новый тип. 

\item \emph{Развёрткой} (unfold) мы получаем из произвольного типа
значение данного рекурсивного типа. Мы словно разворачиваем 
его из значения, этот процесс очень похож на ленивые
вычисления. 

\end{itemize}

Мы узнал некоторые стандартные функции структурной
рекурсии: \In{cond} или \In{if}-выражения, \In{maybe},
\In{foldr}, \In{unfoldr}.

\section{Упражнения}


\begin{itemize}

\item Определите развёртку для деревьев из модуля \In{Data.Tree}. 

\item Определите с помощью свёртки следующие функции:

\begin{code}
sum, prod  :: Num a => [a] -> a
or,  and   :: [Bool] -> Bool
length     :: [a] -> Int
cycle       

unzip      :: [(a,b)] -> ([a],[b])
unzip3     :: [(a,b,c)] -> ([a],[b],[c])
\end{code}


\item Определите с помощью развёртки следующие функции:

\begin{code}
infinity    :: Nat
map         :: (a -> b) -> [a] -> [b]
iterateTree :: (a -> [a]) -> a -> Tree a
zipTree     :: Tree a -> Tree b -> Tree (a, b)
\end{code}

\item Поэкспериментируйте в интерпретаторе с только 
    что определёнными функциями и теми функциями, что мы 
    определяли в этой главе.

\item Рассмотрим ещё один стандартный тип. Он определён в \In{Prelude}.
Это тип \In{Either} (дословно -- один из двух). Этот тип
принимает два параметра:

\begin{code}
data Either a b = Left a | Right b
\end{code}

Значение может быть либо значением типа \In{a}, либо
значением типа \In{b}. Часто этот тип используют как 
\In{Maybe} с информацией об ошибке. Конструктор \In{Left}
хранит сообщение об ошибке, а конструктор \In{Right} 
значение, если его удалось вычислить.

Например мы можем сделать такие определения:

\begin{code}
headSafe :: [a] -> Either String a
headSafe []     = Left "Empty list" 
headSafe (x:_)  = Right x

divSafe :: Fractional a => a -> a -> Either String a
divSafe a 0 = Left "division by zero"
divSafe a b = Right (a/b)
\end{code}

Для этого типа также определена функция свёртки она
называется \In{either}. Не подглядывая в \In{Prelude},
определите её. 

\item Список является частным случаем дерева. Список это
    дерево, в каждом узле которого, лишь однин дочерний узел.
    Деревья из модуля \In{Data.Tree} похожи на списки, 
    но есть в них одно существенное отличие. Они всегда
    содержат хотя бы один элемент. Пустой список не может
    быть представлен в виде такого дерева. Например это 
    различие сказывается, еслим вы захотите определить 
    функцию-аналог \In{takeWhile} для деревьев. 

    Определите деревья, которые не страдают от этого
    недостатка. Определите для них функции свёртки/развёртки,
    а также функции, которые мы определили для стандартных
    деревьев. Определите функцию \In{takeWhile} (в рекурсивном
    виде и в виде развёртки)
    и сделайте их экземпляром класса \In{Monad}, похожий
    на экземпляр для списков.

\end{itemize}

