\chapter{Категориальные типы}

В этой главе мы узнаем как в теории категорий определяются типы. В
теории категорий типы определяются как начальные и конечные объекты в
специальных категориях, которые называются алгебрами функторов. Для
понимания этой главы хорошо освежить в памяти главу о структурной
рекурсии, там где мы говорили о свёртках и развёртках.

\section{Программирование в стиле оригами}

Оригами -- состоит из двух слов ``свёртка'' и ``бумага''. При
программировании в стиле оригами все функции строятся через функции
свёртки и развёртки. Есть даже такие языки программрования, в которых
это единственный способ определения рекурсии. Этот стиль очень хорошо
подходит для ленивых языков программирования, поскольку в связке:

\begin{verbatim}
fold f . unfold g
\end{verbatim}

\noindent 

функции свёртки и развёртки работают синхронно. Функция развёртки не
производит новых элементов до тех пор пока они не понадобятся во внешней
функции свёртки.

Помните в одной из глав мы говорили о том, что рекурсивные функции можно
определять через функцию \texttt{fix}.\\Например так выглядит
рекурсивная функция сложения всех чисел от одного до \texttt{n}:

\begin{verbatim}
sumInt :: Int -> Int
sumInt 0 = 0
sumInt n = n + sumInt (n-1)
\end{verbatim}

Эту функцию мы можем переписать с помощью функции \texttt{fix}. При
вычислении \texttt{fix f} будет составлено значение

\begin{verbatim}
f (f (f (f ...)))
\end{verbatim}

Теперь перепишем функцию \texttt{sumInt} через \texttt{fix}:

\begin{verbatim}
sumInt = fix $ \f n ->
    case n of 
        0   -> 0
        n   -> n + f (n - 1)    
\end{verbatim}

Смотрите лямбда функция в аргументе \texttt{fix} принимает функцию и
число, а возвращает число. Тип этой функции
\texttt{(Int -\textgreater{} Int) -\textgreater{} (Int -\textgreater{} Int)}.
После применения функции \texttt{fix} мы как раз и получим функцию типа
\texttt{Int -\textgreater{} Int}. В лямбда функции рекурсивный вызов был
заменён на вызов функции-параметра \texttt{f}.

Оказывается, что этот приём может быть применён и для рекурсивных типов
данных. Мы можем создать обобщённый тип, который обозначает рекурсивный
тип:

\begin{verbatim}
newtype Fix f = Fix { unFix :: f (Fix f) }
\end{verbatim}

В этой записи мы получаем уравнение неподвижной точки
\texttt{Fix f = f (Fix f)}, где \texttt{f} это некоторый тип с
параметром. Определим тип целых чисел:

\begin{verbatim}
data N a = Zero | Succ a

type Nat = Fix N
\end{verbatim}

Теперь создадим несколько конструкторов:

\begin{verbatim}
zero :: Nat
zero = Fix Zero

succ :: Nat -> Nat
succ = Fix . Succ
\end{verbatim}

Сохраним эти определения в модуле \texttt{Fix.hs} и посмотрим в
интерпретаторе на значения и их типы, ghc не сможет вывести экземпляр
\texttt{Show} для типа \texttt{Fix}, потому что он зависит от типа с
параметром, а не от конкретного типа. Для решения этой проблемы нам
придётся определить экземпляры вручную и подключить несколько расширений
языка. Помните в главе о ленивых вычислениях мы подключали расширение
\texttt{BangPatterns}? Нам понадобятся:

\begin{verbatim}
{-# Language FlexibleContexts, UndecidableInstances #-}
\end{verbatim}

Теперь определим экземпляры для \texttt{Show} и \texttt{Eq}:

\begin{verbatim}
instance Show (f (Fix f)) => Show (Fix f) where
    show x = "(" ++ show (unFix x) ++ ")"

instance Eq (f (Fix f)) => Eq (Fix f) where
    a == b = unFix a == unFix b
\end{verbatim}

Определим списки-оригами:

\begin{verbatim}
data L a b = Nil | Cons a b
    deriving (Show)

type List a = Fix (L a)

nil :: List a
nil = Fix Nil

infixr 5 `cons`

cons :: a -> List a -> List a
cons a = Fix . Cons a
\end{verbatim}

В типе \texttt{L} мы заменили рекурсивный тип на параметр. Затем в
записи \texttt{List a = Fix (L a)} мы производим замыкание по параметру.
Мы бесконечно вкладываем тип \texttt{L a} во второй параметр. Так
получается рекурсивный тип для списков. Составим какой-нибудь список:

\begin{verbatim}
*Fix> :r
[1 of 1] Compiling Fix              ( Fix.hs, interpreted )
Ok, modules loaded: Fix.
*Fix> 1 `cons` 2 `cons` 3 `cons` nil
(Cons 1 (Cons 2 (Cons 3 (Nil))))
\end{verbatim}

Спрашивается, зачем нам это нужно? Зачем нам записывать рекурсивные типы
через тип \texttt{Fix}? Оказывается при такой записи мы можем построить
универсальные функции \texttt{fold} и \texttt{unfold}, они будут
работать для любого рекурсивного типа.

Помните как мы составляли функции свёртки? Мы строили воображаемый
класс, в котором сворачиваемый тип заменялся на параметр. Например для
списка мы строили свёртку так:

\begin{verbatim}
class [a] b where
    (:) :: a -> b -> b
    []  :: b
\end{verbatim}

После этого мы легко получали тип для функции свёртки:

\begin{verbatim}
foldr :: (a -> b -> b) -> b -> ([a] -> b)
\end{verbatim}

Она принимает методы воображаемого класса, в котором тип записан с
параметром, а возвращает функцию из рекурсивного типа в тип параметра.

Сейчас мы выполняем эту процедуру замены рекурсивного типа на параметр в
обратном порядке. Сначала мы строим типы с параметром, а затем получаем
из них рекурсивные типы с помощью конструкции \texttt{Fix}. Теперь
методы класса с параметром это наши конструкторы исходных классов, а
рекурсивный тип записан через \texttt{Fix}. Если мы сопоставим два
способа, то мы сможем получить такой тип для функции свёртки:

\begin{verbatim}
fold :: (f b -> b) -> (Fix f -> b)
\end{verbatim}

Смотрите функция свёртки по-прежнему принимает методы воображаемого
класса с параметром, но теперь класс перестал быть воображаемым, он стал
типом с параметром. Результатом функции свёртки будет функция из
рекурсивного типа \texttt{Fix f} в тип параметр.

Аналогично строится и функция \texttt{unfold}:

\begin{verbatim}
unfold :: (b -> f b) -> (b -> Fix f)
\end{verbatim}

В первой функции мы указываем один шаг разворачивания рекурсивного типа,
а функция развёртки рекурсивно распространяет этот один шаг на
потенциально бесконечную последовательность применений этого одного
шага.

Теперь давайте определим эти функции. Но для этого нам понадобится от
типа \texttt{f} одно свойство. Он должен быть функтором, опираясь на это
свойство, мы будем рекурсивно обходить этот тип.

\begin{verbatim}
fold :: Functor f => (f a -> a) -> (Fix f -> a)
fold f = f . fmap (fold f) . unFix
\end{verbatim}

Проверим эту функцию по типам. Для этого нарисуем схему композиции:

\begin{centering}

\tikzset{commutative diagrams/row sep/normal=1.7cm}
\tikzset{commutative diagrams/column sep/normal=2.0cm}

\begin{tikzcd}
\texttt{Fix f} \rar{\texttt{f}} 
    & \texttt{f (Fix f)} \rar{\texttt{fmap (fold f)}}
    & \texttt{f a} \rar{\texttt{f}} & \texttt{a} 
\end{tikzcd}



\end{centering}

Сначала мы разворачиваем обёртку \texttt{Fix} и получаем значение типа
\texttt{f (Fix f)}, затем с помощью \texttt{fmap} мы внутри типа
\texttt{f} рекурсивно вызываем функцию свёртки и в итоге получаем
значение \texttt{f a}, на последнем шаге мы выполняем свёртку на текущем
уровне вызовом функции \texttt{f}.

Аналогично определяется и функция \texttt{unfold}. Только теперь мы
сначала развернём первый уровень, затем рекурсивно вызовем развёртку
внутри типа \texttt{f} и только в самом конце завернём всё в тип
\texttt{Fix}:

\begin{verbatim}
unfold :: Functor f => (a -> f a) -> (a -> Fix f)
unfold f = Fix . fmap (unfold f) . f
\end{verbatim}

Схема композиции:

\begin{centering}

\tikzset{commutative diagrams/row sep/normal=1.7cm}
\tikzset{commutative diagrams/column sep/normal=2.0cm}

\begin{tikzcd}
\texttt{Fix f} & \texttt{f (Fix f)} \lar[swap]{\texttt{Fix}} 
& \texttt{f a} \lar[swap]{\texttt{fmap (unold f)}} 
& \texttt{a} \lar[swap]{\texttt{f}}
\end{tikzcd}



\end{centering}

Возможно вы уже догадались о том, что функция \texttt{fold} дуальна по
отношению к функции \texttt{unfold}, это особенно наглядно отражается на
схеме композиции. При переходе от \texttt{fold} к \texttt{unfold} мы
просто перевернули все стрелки заменили разворачивание типа \texttt{Fix}
на заворачивание в \texttt{Fix}.

Определим несколько функций для натуральных чисел и списков в стиле
оригами. Для начала сделаем \texttt{L} и \texttt{N} экземпляром класса
\texttt{Functor}:

\begin{verbatim}
instance Functor N where
    fmap f x = case x of
        Zero    -> Zero
        Succ a  -> Succ (f a)

instance Functor (L a) where
    fmap f x = case x of
        Nil         -> Nil
        Cons a b    -> Cons a (f b)
\end{verbatim}

Это всё что нам нужно для того чтобы начать пользоваться функциями
свёртки и развёртки! Определим экземпляр \texttt{Num} для натуральных
чисел:

\begin{verbatim}
instance Num Nat where
    (+) a = fold $ \x -> case x of
            Zero    -> a
            Succ x  -> succ x

    (*) a = fold $ \x -> case x of
            Zero    -> zero
            Succ x  -> a + x

    fromInteger = unfold $ \n -> case n of
            0   -> Zero
            n   -> Succ (n-1)

    abs = undefined
    signum = undefined
\end{verbatim}

Сложение и умножение определены через свёртку, а функция построения
натурального числа из числа типа \texttt{Integer} определена через
развёртку. Сравните с теми функциями, которые мы писали в главе про
структурную рекурсию. Теперь мы не передаём отдельно две функции, на
которые мы будем заменять конструкторы. Эти функции закодированы в типе
с параметром. Для того чтобы этот код заработал нам придётся добавить
ещё одно расширение \texttt{TypeSynonymInstances} наши рекурсивные типы
являются синонимами, а не новыми типами. В рамках стандарта Haskell мы
можем определять экземпляры только для новых типов, для того чтобы
обойти это ограничение мы добавим ещё одно расширение.

\begin{verbatim}
*Fix> succ $ 1+2
(Succ (Succ (Succ (Succ (Zero)))))
*Fix> ((2 * 3) + 1) :: Nat
(Succ (Succ (Succ (Succ (Succ (Succ (Succ (Zero))))))))
*Fix> 2+2 == 2*(2::Nat)
True
\end{verbatim}

Определим функции на списках. Для начала определим две вспомогательные
функции, которые извлекают голову и хвост списка:

\begin{verbatim}
headL :: List a -> a
headL x = case unFix x of
    Nil         -> error "empty list"
    Cons a _    -> a

tailL :: List a -> List a
tailL x = case unFix x of
    Nil         -> error "empty list"
    Cons a b    -> b
\end{verbatim}

Теперь определим несколько новых функций:

\begin{verbatim}
mapL :: (a -> b) -> List a -> List b
mapL f = fold $ \x -> case x of
    Nil         -> nil
    Cons a b    -> f a `cons` b

takeL :: Int -> List a -> List a
takeL = curry $ unfold $ \(n, xs) -> 
    if n == 0 then Nil
              else Cons (headL xs) (n-1, tailL xs)
\end{verbatim}

Сравните эти функции с теми, что мы определяли в главе о структурной
рекурсии. Проверим работают ли эти функции:

\begin{verbatim}
*Fix> :r
[1 of 1] Compiling Fix              ( Fix.hs, interpreted )
Ok, modules loaded: Fix.
*Fix> takeL 3 $ iterateL (+1) zero
(Cons (Zero) (Cons (Succ (Zero)) (Cons (Succ (Succ (Zero))) (Nil))))
*Fix> let x = 1 `cons` 2 `cons` 3 `cons` nil
*Fix> mapL (+10) $ x `concatL` x
(Cons 11 (Cons 12 (Cons 13 (Cons 11 (Cons 12 (Cons 13 (Nil)))))))
\end{verbatim}

Обратите внимание, на то что с большими буквами мы пишем \texttt{Cons} и
\texttt{Nil} когда хотим закодировать функции для свёртки-развёртки, а с
маленькой буквы пишем значения рекурсивного типа. Надеюсь, что вы
разобрались на примерах как устроены функции \texttt{fold} и
\texttt{unfold}, потому что теперь мы перейдём к теории, которая за этим
стоит.

\section{Индуктивные и коиндуктивные типы}

С точки зрения теории категорий функция свёртки является катаморфизмом,
а функция развёртки -- анаморфизмом. Напомню, что катаморфизм -- это
функция которая ставит в соответствие объектам категории с начальным
объектом стрелки, которые начинаются из начального объекта, а
заканчиваются в данном объекте. Анаморфизм -- это перевёрнутый наизнанку
катаморфизм.

Начальным и конечным объектом будет рекурсивный тип. Вспомним тип
свёртки:

\begin{verbatim}
fold :: Functor f => (f a -> a) -> (Fix f -> a)
\end{verbatim}

Функция свёртки строит функции, которые ведут из рекурсивного типа в
произвольный тип, поэтому в данном случае рекурсивный тип будет
начальным объектом. Функция развёртки строит из произвольного типа
данный рекурсивный тип, на языке теории категорий она строит стрелку из
произвольного объекта в рекурсивный, это означает что рекурсивный тип
будет конечным объектом.

\begin{verbatim}
unfold :: Functor f => (a -> f a) -> (a -> Fix f)
\end{verbatim}

Категории, которые определяют рекурсивные типы таким образом называются
(ко)алгебрами функторов. Видите в типе и той и другой функции стоит
требование о том, что \texttt{f} является функтором. Катаморфизм и
анаморфизм отображают объекты в стрелки. По типу функций \texttt{fold} и
\texttt{unfold} мы можем сделать вывод, что объектами в нашей категории
будут стрелки вида

\begin{verbatim}
f a -> a
\end{verbatim}

или для свёрток:

\begin{verbatim}
a -> f a
\end{verbatim}

А стрелками будут обычные функции одного аргумента. Теперь дадим более
формальное определение.

Эндофунктор $F : \CatA \Ra \CatA$ определяет стрелки
$\alpha : FA \Ra A$, которые называется $F$-\emph{алгебрами}. Стрелку
$h : A \Ra B$ называют $F$-\emph{гомоморфизмом}, если следующая
диаграмма коммутирует:

\begin{centering}

\begin{tikzcd}
FA \rar{\alpha} \dar[swap]{Fh} & A \dar{h} \\
FB \rar[swap]{\beta}          & B 
\end{tikzcd}



\end{centering}

Или можно сказать по другому, для $F$-алгебр $\alpha:FA \Ra A$ и
$\beta : FB \Ra B$ выполняется:

\[Fh \Co \beta = \alpha \Co h\]

Это свойство совпадает со свойством естественного преобразования только
вместо одного из функторов мы подставили тождественный функтор $I$.
Определим категорию $\Alg$, для категории $\CatA$ и эндофунктора
$F : \CatA \Ra \CatA$

\begin{itemize}
\item
  Объектами являются $F$-алгебры $FA \Ra A$, где $A$ -- объект категории
  $\CatA$
\item
  Два объекта $\alpha : FA \Ra A$ и $\beta : FB \Ra B$ соединяет
  $F$-гомоморфизм $h : A \Ra B$. Это такая стрелка из $\CatA$, для
  которой выполняется:
\end{itemize}

\[Fh \Co \beta = \alpha \Co h\]

\begin{itemize}
\item
  Композиция и тождественная стрелка взяты из категории $\CatA$.
\end{itemize}

Если в этой категории есть начальный объект $in_F : FT \Ra T$, то
определён катаморфизм, который переводит объекты $FA \Ra A$ в стрелки
$T \Ra A$. Причём следующая диаграмма коммутирует:

\begin{centering}


\begin{tikzcd}
FT \rar{in_F} \dar[swap]{F \cata{\alpha}} & T \dar{\cata{\alpha}} \\
FA \rar[swap]{\alpha}          & A
\end{tikzcd}



\end{centering}

Этот катаморфизм и будет функцией свёртки для рекурсивного типа $Т$.
Понятие $\Alg$ можно перевернуть и получить категорию $\CoAlg$.

\begin{itemize}
\item
  Объектами являются $F$-коалгебры $A \Ra FA$, где $A$ -- объект
  категории $\CatA$
\item
  Два объекта $\alpha : FA \Ra A$ и $\beta : FB \Ra B$ соединяет
  $F$-когомоморфизм \mbox{$h : A \Ra B$}. Это такая стрелка из $\CatA$,
  для которой выполняется:

  \[h \Co \alpha = \beta \Co Fh\]
\item
  Композиция и тождественная стрелка взяты из категории $\CatA$.
\end{itemize}

Если в этой категории есть конечный объект, его называют
$out_F : T \Ra FT$, то определён анаморфизм, который переводит объекты
$A \Ra FA$ в стрелки $A \Ra T$.\\Причём следующая диаграмма коммутирует:

\begin{centering}


\newcommand{\Ra}[0]{\rightarrow}
\newcommand{\RA}[0]{\Rightarrow}
\newcommand{\Or}[0]{\ |\ }
\newcommand{\Br}[1]{\{#1\} }


%------------------------------
% category
\newcommand{\Co}{\footnotesize\textbf{;}\normalsize}
\newcommand{\CoT}{\textbf{~$;_{T}$~} }
\newcommand{\Ha}{\In{H}}
\newcommand{\CatA}{\mathcal{A}}
\newcommand{\CatB}{\mathcal{B}}
\newcommand{\cata}[1]{\llparenthesis\,#1\,\rrparenthesis}
\newcommand{\ana}[1]{[\hspace{-2.2pt}(\,#1\,)\hspace{-2.2pt}]}

\newcommand{\Alg}{\textbf{Alg}(F)} 
\newcommand{\CoAlg}{\textbf{CoAlg}(F)} 


\begin{tikzcd}
T \rar{in_F} \dar[swap]{\ana{\alpha}} & FT \dar{F \ana{\alpha}} \\
A \rar[swap]{\alpha}          & FA
\end{tikzcd}



\end{centering}

Если для категории $\CatA$ и функтора $F$ определены стрелки $in_F$ и
$out_F$, то они являются взаимнообратными и определяют изоморфизм
$T \cong FT$. Часто объект $T$ в случае $\Alg$ обозначают $\mu_F$,
поскольку начальный объект определяется функтором $F$, а в случае
$\CoAlg$ обозначают $\nu_F$.

Типы, которые являются начальными объектами, принято называть
индуктивными, а типы, которые являются конечными объектами --
коиндуктивными.

\subsection{Существование начальных и конечных объектов}

Мы говорили, что если начальный(конечный) объект существует, а когда он
существует? Рассмотрим один важный случай. Если категория является
категорией, в которой объектами являются полные частично упорядоченные
множества, а стрелками являются монотонные функции, такие категории
называют $\textbf{CPO}$, и функтор -- полиномиальный, то начальный и
конечный объекты существуют.

\subsubsection{Полные частично упорядоченные множества}

Оказывается на значениях можно ввести частичный порядок. Порядок
называется частичным, если отношение $\leq$ определено не для всех
элементов, а лишь для некоторых из них. Частичный порядок на значениях
отражает степень неопределённости значения. Самый маленький объект это
полностью неопределённое значение $\bot$. Любое значение типа содержит
больше определённости чем $\bot$.

Для того чтобы не путать упорядочивание значений по степени
определённости с обычным числовым порядком, пользуются специальным
символом $\sqsubseteq$. Запись

\[a \sqsubseteq b\]

\noindent 

означает, что $b$ более определено (или информативнее) чем $a$.

Так для логических значений определены два нетривиальных сравнения:

\[data\ Bool\ =\ True \Or False\]

\[\bot \sqsubseteq True\] \[\bot \sqsubseteq False\]

Мы будем называть нетривиальными сравнения в которых, компоненты слева и
справа от $\sqsubseteq$ не равны. Например ясно, что
$True \sqsubseteq True$ или $\bot \sqsubseteq \bot$. Это тривиальные
сравнения и мы их будем лишь подразумевать. Считается, что если два
значения определены полностью, то мы не можем сказать какое из них
информативнее. Так к примеру для логических значений мы не можем сказать
какое значение более определено $True$ или $False$.

Рассмотрим пример по-сложнее. Частично определённые значения:

\[data\ Maybe\ a = Nothing \Or Just\ a\]

\[\begin{array}{l@{\ \sqsubseteq \ }l}
    \bot & Nothing \\
    \bot & Just\ \bot \\
    \bot & Just\ a \\
    Just\ a & Just\ b,\qquad \text{если } a \sqsubseteq b    
\end{array}\]

Если вспомнить как происходит вычисление значения, то значение $a$ менее
определено чем $b$, если взрывное значение $\bot$ в $a$ находится ближе
к корню значения, чем в $b$. Итак получается, что в категории
$\textbf{Hask}$ объекты это множества с частичным порядком. Что означает
требование монотонности функции?\\Монотонность в контексте операции
$\sqsubseteq$ говорит о том, что чем больше определён вход функции тем
больше определён выход:

\[a \sqsubseteq b \quad \RA \quad f\ a \sqsubseteq f\ b\]

Это требование накладывает запрет на возможность проведения
сопоставления с образцом по значению $\bot$. Иначе мы можем определять
немонотонные функции вроде:

\begin{verbatim}
isBot :: Bool -> Bool
isBot undefined = True
isBot _         = undefined
\end{verbatim}

Полнота частично упорядоченного множества означает, что у любой
последовательности $x_n$

\[x_0 \sqsubseteq x_1 \sqsubseteq x_2 \sqsubseteq ...\]

\noindent 

есть значение $x$, к которому она сходится. Это значение называют
супремумом множества. Что такое полные частично упорядоченные множества
мы разобрались. А что такое полиномиальный функтор?

\subsubsection{Полиномиальный функтор}

Полиномиальный функтор -- это функтор который построен лишь с помощью
операций суммы, произведения, постоянных функторов, тождественного
фуктора и композиции функторов. Определим эти операции:

\begin{itemize}
\item
  Сумма функторов $F$ и $G$ определяется через операцию суммы объектов:

  \[(F+G)X = FX + GX\]
\item
  Произведение функторов $F$ и $G$ определяется через операцию
  произведения объектов:

  \[(F\times G)X = FX \times GX\]
\item
  Постоянный функтор отображает все объекты категории в один объект, а
  стрелки в тождественнубю стрелку этого объекта, мы будем обозначать
  постоянный функтор подчёркиванием:
\end{itemize}

\begin{eqnarray*}
    \underline{A}X &=& A \\
    \underline{A}f &=& id_A \\
\end{eqnarray*}

\begin{itemize}
\item
  Тождественный функтор оставляет объекты и стрелки неизменными:
\end{itemize}

\begin{eqnarray*}
    IX &=& X \\
    If &=& f \\
\end{eqnarray*}

\begin{itemize}
\item
  Композиция функторов $F$ и $G$ это последовательное применение
  функторов

  \[FGX = F(GX)\]
\end{itemize}

По определению функции построенные с помощью этих операций называют
полиномиальными. Определим несколько типов данных с помощью
полиномиальных функторов. Определим логические значения:

\[Bool = \mu(\underline{1} + \underline{1})\]

Объект $1$ обозначает любую константу, это конечный объект исходной
категории. Нам не важны имена конструкторов, но важна структура типа.
$\mu$ обозначает начальный объект в $F$-алгебре.

Определим натуральные числа:

\[Nat = \mu(\underline{1} + I)\]

Эта запись обозначает начальный объект для $F$-алгебры с функтором
$F=\underline{1}+I$. Посмотрим на определение списка:

\[List_A = \mu(\underline{1} + \underline{A} \times I)\]

Список это начальный объект $F$-алгебры
$\underline{1}+\underline{A}\times I$. Также можно определить бинарные
деревья:

\[BTree_A = \mu(\underline{A} + I \times I )\]

Определим потоки:

\[Stream_A = \nu (\underline{A} \times I)\]

Потоки являются конечным объектом $F$-коалгебры, где
$F= \underline{A} \times I$.

\section{Гиломорфизм}

Оказывается, что с помощью катаморфизма и анаморфизма мы можем
определить функцию \texttt{fix}, т.е.\textasciitilde{}мы можем выразить
любую рекурсивную функцию с помощью структурной рекурсии.

Функция \texttt{fix} строит бесконечную последовательность применений
некоторой функции \texttt{f}.

\begin{verbatim}
f (f (f ...)))
\end{verbatim}

Сначала с помощью анаморфизма мы построим бесконечный список, который
содержит функцию \texttt{f} во всех элементах:

\begin{verbatim}
repeat f = f : f : f : ...
\end{verbatim}

А затем заменим конструктор \texttt{:} на применение. В итоге мы получим
такую функцию:

\begin{verbatim}
fix :: (a -> a) -> a
fix = foldr ($) undefined . repeat 
\end{verbatim}

Убедимся, что эта функция работает:

\begin{verbatim}
Prelude> let fix = foldr ($) undefined . repeat
Prelude> take 3 $ y (1:)
[1,1,1]
Prelude> fix (\f n -> if n==0 then 0 else n + f (n-1)) 10
55
\end{verbatim}

Теперь давайте определим функцию \texttt{fix} через функции
\texttt{cata} и \texttt{ana}:

\begin{verbatim}
fix :: (a -> a) -> a
fix = cata (\(Cons f a) -> f a) . ana (\a -> Cons a a)
\end{verbatim}

Эта связка анаморфизм с последующим катаморфизмом встречается так часто,
что ей дали специальное имя. \emph{Гиломорфизмом} называют функцию:

\begin{verbatim}
hylo :: Functor f => (f b -> b) -> (a -> f a) -> (a -> b) 
hylo phi psi = cata phi . ana psi
\end{verbatim}

Отметим, что эту функцию можно выразить и по-другому:

\begin{verbatim}
hylo :: Functor f => (f b -> b) -> (a -> f a) -> (a -> b) 
hylo phi psi = phi . (fmap $ hylo phi psi) . psi
\end{verbatim}

Этот вариант более эффективен по расходу памяти, мы не строим
промежуточное значение \texttt{Fix f}, а сразу обрабатываем значения в
функции \texttt{phi} по ходу их построения в функции \texttt{psi}.
Давайте введём инфиксную операцию гиломорфизм для этого определения:

\begin{verbatim}
(>>) :: Functor f => (a -> f a) -> (f b -> b) -> (a -> b) 
psi >> phi = phi . (fmap $ hylo phi psi) . psi
\end{verbatim}

Теперь давайте скроем одноимённую функцию из \texttt{Prelude} и
определим несколько рекурсивных функций с помощью гиломорфизма. Начнём с
функции вычисления суммы чисел от нуля до данного числа:

\begin{verbatim}
sumInt :: Int -> Int
sumInt = range >> sum
    
sum x = case x of
    Nil      -> 0 
    Cons a b -> a + b

range n 
    | n == 0    = Nil 
    | otherwise = Cons n (n-1)
\end{verbatim}

Сначала мы создаём в функции \texttt{range} список всех чисел от данного
числа до нуля. А затем в функции \texttt{sum} складываем значения.
Теперь мы можем легко определить функцию вычисления факториала:

\begin{verbatim}
fact :: Int -> Int
fact = range >> prod
    
prod x = case x of
    Nil      -> 1 
    Cons a b -> a * b
\end{verbatim}

Напишем функцию, которая извлекает из потока n-тый элемент. Сначала
определим тип для потока:

\begin{verbatim}
type Stream a = Fix (S a)

data S a b = a :& b
    deriving (Show, Eq)

instance Functor (S a) where
    fmap f (a :& b) = a :& f b  


headS :: Stream a -> a
headS x = case unFix x of
    (a :& _) -> a


tailS :: Stream a -> Stream a
tailS x = case unFix x of
    (_ :& b) -> b
\end{verbatim}

Теперь функцию извлечения элемента:

\begin{verbatim}
getElem :: Int -> Stream a -> a
getElem = curry (enum >> elem) 
    where elem ((n, a) :& next) 
                | n == 0    = a
                | otherwise = next
          enum (a, st) = (a, headS st) :& (a-1, tailS st)
\end{verbatim}

В функции \texttt{enum} мы добавляем к элементам потока убывающую
последовательность чисел, она стартует из данного числа. Элемент,
который нам нужен, будет содержать в этой последовательности число ноль.
В функции \texttt{elem} мы как раз и извлекаем тот элемент рядом с
которым хранится число ноль. Обратите внимание на то, что рекурсия
встроена в этот алгоритм, если данное число не равно нулю, мы просто
извлекаем следующий элемент.

С помощью этой функции мы можем вычислить n-тое число из ряда чисел
Фибоначчи. Сначала создадим поток чисел Фибоначчи:

\begin{verbatim}
fibs :: Stream Int
fibs = ana (\(a, b) -> a :& (b, a+b)) (0, 1)
\end{verbatim}

Теперь просто извлечём n-тый элемент из потока чисел Фибоначчи:

\begin{verbatim}
fib :: Int -> Int
fib = flip getElem fibs
\end{verbatim}

Вычислим поток всех простых чисел. Мы будем вычислять его по алгоритму
``решето Эратосфена''. В начале алгоритма у нас есть поток целых чисел и
известно, что первое число является простым.

2, 3, 4, 5, 6, 7, 8, 9, 10, 11, 12, 13, 14, 15 \ldots{}

В процессе этого алгоритма мы вычёркиваем все не простые числа. Сначала
мы ищем первое не зачёркнутое число и помещаем его в результирующий
поток, а на следующий шаг алгоритма мы передаём исходный, поток в
котором зачёркнуты все числа кратные тому, что мы положили последним:

2

3, \sout{4}, 5, \sout{6}, 7, \sout{8}, 9, \sout{10}, 11, \sout{12}, 13,
\sout{14}, 15, \ldots{}

Теперь мы ищем первое незачёркнутое число и помещаем его в результат. А
на следующий шаг рекусии передаём поток, в котором зачёркнуты все числа
кратные новому простому числу:

2, 3

\sout{4}, 5, \sout{6}, 7, \sout{8}, \sout{9}, \sout{10}, \sout{12}, 13,
\sout{14}, \sout{15}, \ldots{}

И так далее, на каждом шаге мы будем получать одно простое число.
Зачёркивание мы будем имитировать с помощью типа \texttt{Maybe}. Всё
начинается с потока целых чисел, в котором не зачёркнуто ни одно число:

\begin{verbatim}
nums :: Stream (Maybe Int)
nums = mapS Just $ iterateS (+1) 2 

mapS :: (a -> b) -> Stream a -> Stream b
mapS f = ana $ \xs -> (f $ headS xs) :& tailS xs

iterateS :: (a -> a) -> a -> Stream a
iterateS f = ana $ \x -> x :& f x
\end{verbatim}

В силу ограничений системы типов Haskell мы не можем определить
экземпляр \texttt{Functor} для типа \texttt{Stream}, поскольку
\texttt{Stream} является не самостоятельным типом а типом-синонимом.
Поэтому нам приходится определить функцию \texttt{mapS}. Определим шаг
рекурсии:

\begin{verbatim}
primes :: Stream Int
primes = ana erato nums

erato xs = n :& erase n ys
    where n  = fromJust $ headS xs  
          ys = dropWhileS isNothing xs
\end{verbatim}

Переменная \texttt{n} содержит первое не зачёркнутое число на данном
шаге. Переменная \texttt{ys} указывает на список чисел, из начала
которого удалены все зачёркнутые числа. Функции \texttt{isNothing} и
\texttt{fromJust} взяты из стандартного модуля \texttt{Data.Maybe}. Нам
осталось определить лишь две функции. Это аналог функции
\texttt{dropWhile} на списках. Эта функция удаляет из начала списка все
элементы, которые удовлетворяют некоторому предикату. Вторая функция
\texttt{erase} вычёркивает все числа в потоке кратные данному.

\begin{verbatim}
dropWhileS :: (a -> Bool) -> Stream a -> Stream a
dropWhileS p = psi >> phi 
    where phi ((b, xs) :& next) = if b then next else xs
          psi xs = (p $ headS xs, xs) :& tailS xs
\end{verbatim}

В этой функции мы сначала генерируем список пар, который содержит
значения предиката и остатки списка, а затем находим в этом списке
первый такой элемент, значение которого равно \texttt{False}.

\begin{verbatim}
erase :: Int -> Stream (Maybe a) -> Stream (Maybe a)
erase n xs = ana phi (0, xs)
    where phi (a, xs) 
            | a == 0    = Nothing  :& (a', tailS xs)
            | otherwise = headS xs :& (a', tailS xs)
            where a' = if a == n-1 then 0 else (a+1)
\end{verbatim}

В функции \texttt{erase} мы заменяем на \texttt{Nothing} каждый элемент,
порядок следования которого кратен аргументу \texttt{n}. Проверим, что у
нас получилось:

\begin{verbatim}
*Fix> primes 
(2 :& (3 :& (5 :& (7 :& (11 :& (13 :& (17 :& (19 :& (23 :& 
(29 :& (31 :& (37 :& (41 :& (43 :& (47 :& (53 :& (59 :& 
(61 :& (67 :& (71 :& (73 :& (79 :& (83 :& (89 :& (97 :& 
(101 :& (103 :& (107 :& (109 :& (113 :& (127 :& (131 :&
...
\end{verbatim}

\section{Краткое содержание}

В этой главе мы узнали, что любая рекурсивная функция может быть
выражена через структурную рекурсию. Мы узнали как в теории категорий
определяются типы. Типы являются начальными и конечными объектами в
специальных категориях, которые называются алгебрами функторов. Слоган
теории категорий гласит:

\begin{quote}
Управляющие структуры определяются структурой типов.
\end{quote}

Определив тип, мы получаем вместе с ним две функции структурной
рекурсии, это катаморфизм (для начальных объектов) и анаморфизм (для
конечных объектов). С помощью катаморфизма мы можем сворачивать значение
данного типа в значения любого другого типа, а с помощью анаморфизма мы
можем разворачивать значения данного типа из значений любого другого
типа. Также мы узнали, что категория $\textbf{Hask}$ является категорией
$\textbf{CPO}$, категорией полных частично упорядоченных множеств.

\section{Упражнения}

\begin{itemize}
\item
  Потренируйтесь в определении рекурсивных функций через гиломорфизм.
  Попробуйте переписать как можно больше определений из главы о
  структурной рекурсии в терминах типа \texttt{Fix} и функций
  \texttt{cata}, \texttt{ana} и \texttt{hylo}. Также потренируйтесь на
  стандартных функциях из модуля \texttt{Prelude}. Определите новые типы
  через \texttt{Fix} например деревья из модуля \texttt{Data.Tree}.
  Попробуйте свои силы на функциях по-сложнее например алгоритме
  эвристического поиска.
\item
  Определите монадные версии рекурсивных функций:

\begin{verbatim}
cataM :: (Monad m, Traversable t) => (t a -> m a) -> Fix t -> m a
anaM  :: (Monad m, Traversable t) => (a -> m (t a)) -> (a -> m (Fix t))

hyloM :: (Monad m, Traversable t) => (t b -> m b) -> (a -> m (t a)) -> (a -> m b)
\end{verbatim}

  С помощью этих функций мы, например, можем преобразовывать дерево
  выражения и при этом обновлять какое-нибудь состояние или читать из
  общего окружения.

  В этом определении стоит новый класс \texttt{Traversable}. Разберитесь
  с ним самостоятельно. Немного подскажу. Этот класс появился вместе с
  классом \texttt{Applicative}. Когда разработчики поняли о
  существовании полезной абстракции, которая ослабляет класс
  \texttt{Monad}, они также обратили внимание на функцию
  \texttt{sequence}:

\begin{verbatim}
sequence :: Monad m => [m a] -> m [a]
sequence = foldr (liftM2 (:)) (return [])  
\end{verbatim}

  Эту функцию можно записать с помощью одних лишь методов класса
  \texttt{Applicative}. Поэтому ограничение в контексте функции
  избыточно. Класс \texttt{Traversable} предназначени для устранения
  этой неточности. Посмотрим на основной метод класса:

\begin{verbatim}
class (Functor t, Foldable t) => Traversable t where
    traverse :: Applicative f => (a -> f b) -> t a -> f (t b)
\end{verbatim}

  Тип очень похож на тип функции \texttt{mapM}. И не случайно, ведь
  \texttt{mapM} определяется через \texttt{sequence}. Только теперь
  вместо списка стоит более общий тип. Это тип \texttt{Foldable},
  который определяет список как нечто, на чём можно проводить операции
  свёртки.
\end{itemize}
